\chapter{Conclusions}\label{chapter:conclusion}

This chapter serves to summarise the work and contributions of this thesis. Each
chapter contains a detailed summary section, and so the summary here will be
brief.

\section{Research summary}

\section{Contributions}

\section{Complementary Research}

During my PhD I was involved in several research projects which were related to
my PhD research but did not contribute directly to my thesis. The first two
projects focused on the IPD, and more specifically, in using
reinforcement learning to introduce a series of strategies based on different
structures such as finite state machines, hidden Markov models and neural
networks~\cite{Knight2017, Harper2017}.

Another undertaken project included exploring rhino poaching behaviour
using evolutionary game theory~\cite{Glynatsi2018}. Rhino populations are at
critical level today and in protected areas devaluation approaches are used to
secure the life of the animals. The effectiveness of
these approaches, however, relies on poachers behaviour as they can be selective
and not kill devalued rhinos or indiscriminate. Using evolutionary game theory
and ordinary differential equations allowed me to model populations of
differently behaving poachers, and demonstrate that full devaluation of all rhinos is
likely to lead to indiscriminate poaching and that devaluating of rhinos can
only be effective when implemented along with a strong disincentive framework.
The paper aimed to contribute to the necessary research required for an informed
discussion about the lively debate on legalising rhino horn trade.

\section{Future research directions}
