\chapter{Conclusions}\label{chapter:conclusion}

This Chapter serves to summarise the work and contributions of this thesis. Each
chapter contains a detailed chapter summary section, and so the summary here will be
brief.

\section{Research Summary}

The fundamental research question of this thesis has been the same question that
has troubled the scientific community since the formulation of the Iterated
Prisoner's Dilemma in 1950. Namely, what is the optimal behaviour an Iterated
Prisoner's Dilemma strategy should adapt as a response to different
environments.

Chapter~\ref{chapter:introduction} introduced the Iterated Prisoner's Dilemma,
carried out an initial literature review and outlined the research tasks of this
thesis. A more detailed literature review was presented in
Chapter~\ref{chapter:literature_review}. The literature reviewed in
Chapter~\ref{chapter:literature_review} was divided into different research
topics. These included evolutionary dynamics, intelligently designed strategies,
structured strategies and training and software that has been developed
specifically for the game.

In Chapter~\ref{chapter:bibliometric_study} a bespoke research software tool
called \mintinline{python}{arcas} was developed and used to collect a data set
of articles' metadata on the Iterate Prisoner's Dilemma. A topic modelling
technique, called Latent Dirichlet Allocation, was applied to the abstracts of these
articles and allocated them into five different research topics. These were human
subject research, biological studies, strategies, evolutionary dynamics on
networks and modelling problems as a Prisoner's Dilemma.

The bespoke data set was further analysed to explore whether the academic field
of the Prisoner's Dilemma is cooperative and whether there is influence between
the authors. It was shown that the field of the Iterated Prisoner's Dilemma is a
collaborative field, yet it is not necessarily more collaborative than other
fields. Many authors tend to collaborate with authors from one community
and are not involved in multiple communities. The collaborativeness was also
explored over time, and it was shown that since the first publications authors
tended to write only with a single community and that it is not an effect of a
specific time period. Exploring the influence of authors
in the field based on the specific publications showed that authors do not gain
much influence, and the only ones with influence are the ones connected to a
``main'' group.

Chapter~\ref{chapter:meta_tournaments} examined the performance of a collection
of \numberofstrategies strategies in the largest collection of computer
tournaments in the field. The results across the \numberofalltournaments
tournaments of various tournaments types deduced that there was not a single strategy that
performs well in diverse Iterated Prisoner's Dilemma scenarios.
The later parts of the Chapter analysed and extracted the salient features of the best
performing strategies across the various tournament types and established that there
are several properties that heavily influence the best performing
strategies. There were: be nice, be provocable and forgiving, be
a little envious, be clever, and adapt to the environment.

Chapter~\ref{chapter:memory_one} investigated best response memory-one
strategies with a theory of mind. It presented
several theoretical and numerical results. More specifically, it proved that:

\begin{itemize}
    \item The utility of a
    memory-one strategy against a set of memory-one opponents can be written as a
    sum of ratios of quadratic forms.
    \item There is a compact way of identifying
    a memory-one best response to a group of opponents through a search over a
    discrete set.
    \item There is a condition for which in an environment of
    memory-one opponents defection is the stable choice, based only on the
    coefficients of the opponents.
\end{itemize}

Additionally, the numerical results of Chapter~\ref{chapter:memory_one}
reinforced established result of the literature. Namely, they showed that
extortionate play is not always optimal by showing that optimal play is often
not extortionate, and that memory-one strategies suffer from their limited
memory in multi agent interactions and can be out performed by optimised
strategies with longer memory.

Chapter~\ref{chapter:lstm} %TODO include results.

\section{Contributions}

This thesis has made novel contributions across various themes. Numerous
research software packages have been implemented as part of this thesis. These
packages have been written following the highest standards of software
development, and have been made available so that other researchers can
contribute to and use them. The packages include \mintinline{python}{arcas} a tool designed for
scraping academic articles from various APIs and
\mintinline{python}{sequences-sensei} a project for performing genetic
algorithms. Additionally, software contributions were made to well established
Python libraries such as SymPy~\cite{sympy} and Axelrod-Python
Library~\cite{axelrodproject}.

A total of six accompanying data sets have been generated as a result of this thesis,
which include one of the largest collection of Iterated Prisoner's Dilemma
tournaments known to the field:

\begin{enumerate}
    \item Articles' meta data on the Prisoner's Dilemma~\cite{pd_data_2018}.
    \item Articles' meta data on the Price of Anarchy~\cite{anarchy_data_2018}.
    \item Articles' meta data on Auction Games~\cite{auction_data_2018}.
    \item Raw data for: ``Stability of defection, optimisation of strategies and
    the limits of memory in the Prisoner's Dilemma''~\cite{glynatsi2019}.
    \item A data set of 45686 Iterated Prisoner's Dilemma tournaments' results~\cite{data} %TODO include raw data citation, check citation is still data.
    \item Best response sequences in the Prisoner's Dilemma. %TODO include citation
\end{enumerate}

These have been archived and made available via Zenodo, and likewise, are
available to other researchers. They can be used to conduct further analysis and
provide new insights to the field.

A total of four scientific manuscripts presenting the methodology, analysis and
results of this thesis have been prepared and are currently under submission to
respective academic journals. The title of these manuscripts are:

\begin{enumerate}
    \item A bibliometric study of research topics, collaboration and influence
    in the field of the Iterated Prisoner's Dilemma.
    \item A meta analysis of tournaments and an evaluation of performance in the
    Iterated Prisoner's Dilemma.
    \item Recognising and evaluating the effectiveness of extortion in the
    Iterated Prisoner's Dilemma~\cite{Knight2019}.
    \item A theory of mind: Best responses to memory-one strategies.
    The limitations of extortion and restricted memory
\end{enumerate} %TODO add citations

These manuscripts have been uploaded on the pre
print server arXiv and are currently available and accessible to the scientific
community.

Designing new strategies is an important type of research for the field.
This thesis has introduced an abundant number of properties of successful
strategies which can be of interest to researchers designing a new strategy
for new environments, or just to understand the reasons behind some strategies
being better than others. Complementing this, a new mathematical framework has
been developed for the better understanding of memory-one strategies and an
an initial understanding of using recurrent neural networks to train Iterate Prisoner's
Dilemma strategies has been presented.

Finally, this thesis has contributed to the continuous understanding of the emergence of
cooperation by providing a condition for which cooperation can not occur in
memory-one environments. It has also has proven that constrained quadratic
ratio optimisation problems that are non concave can be solved explicitly by
using resultant theory.

% Finally, this thesis has shown the potential for an unsupervised learning approach to training a neural network ... drop previous "Finally"

\section{Complementary Research}

The results of this thesis are not the only scientific results to which I
contributed during this doctoral research. The publications that will be
discussed in this section are publications to which I am an author.

Two other projects which focused on the Iterated
Prisoner's Dilemma have been
\cite{Knight2017, Harper2017}. The works of~\cite{Knight2017, Harper2017}
focused on the usage of reinforcement learning algorithms (genetic algorithms
and particle swarm optimisation algorithms) in training a series of strategies
based on different structures such as finite state machines, hidden Markov
models and neural networks. These strategies were trained in two settings:

\begin{itemize}
    \item A Moran process which is an evolutionary model of invasion and
    resistance across time during which high performing individuals are more
    likely to be replicated.
    \item A standard tournament.
\end{itemize}

The results of~\cite{Knight2017} were confirmed in
Chapter~\ref{chapter:meta_tournaments}. The trained strategies performed at
the top of the standard tournament surpassing well established
strategies such as Tit For Tat, Pavlov, Gradual and zero-determinant strategies.
In~\cite{Harper2017} it was observed that the trained strategies (with no manual
input) evolved the ability to have a handshake, to recognise themselves. This
was particularly important in a Moran process of resisting invasion where a
single individual of another type is introduced and the strategies need to
resist the invasion.

Another undertaken project included exploring rhino poaching behaviour using
evolutionary game theory~\cite{Glynatsi2018}. Rhino populations are at critical
level today and in protected areas devaluation approaches are used to secure the
life of the animals. The effectiveness of these approaches, however, relies on
poachers behaviour as they can be selective and not kill devalued rhinos or
indiscriminate. Populations of differently behaving poachers were modelled using
evolutionary game theory. The results
demonstrated that full devaluation of all rhinos is likely to lead to
indiscriminate poaching and that devaluating of rhinos can only be effective
when implemented along with a strong disincentive framework. The paper aimed to
contribute to the necessary research required for an informed discussion about
the lively debate on legalising rhino horn trade.

Finally, providing science outreach workshops is a great way to gain a deeper
understanding of science and its applications, and enhancing students interest
in science. With that in mind I created an open source educational tutorial, called Game
Theory and Python~\cite{Glynatsi2017_game}, aimed at introducing participants to % TODO Add JOSE paper when it's done
game theory and more specifically to repeated games. The tutorial is
aimed at two groups of individuals: individuals familiar with Python (programmers) who want
to start to learn game theory and mathematicians with little or no programming
knowledge as a pathway to programming through the interesting subject. The
tutorial has gained much interest and is currently under submission at the
Journal of Open Source Education.

A full list of the publications produced during the research presented in this
section is:

\begin{enumerate}
    \item Reinforcement Learning Produces Dominant Strategies for the
    Iterated Prisoner's Dilemma~\cite{Knight2017}.
    \item Evolution Reinforces Cooperation with the Emergence of Self-Recognition
    Mechanisms: an empirical study of the Moran process for the iterated
    Prisoner's dilemma~\cite{Harper2017}.
    \item An Evolutionary Game Theoretic Model of Rhino Horn Devaluation~\cite{Glynatsi2018}.
    \item Game Theory and Python~\cite{Glynatsi2017_game}.
\end{enumerate}

\section{Future research directions}

Each part of this thesis has given rise to further interesting questions and
research directions that, although not in the scope of the current work, would
improve or compliment it.

\textbf{Future research - Meta tournament Analysis}

In Chapter~\ref{chapter:meta_tournaments} during the data collection the
probability of noise was allowed to vary between values of \(0\) and \(1\).
However, it was established that large values of noise (\(>
0.1\)) caused an impactful variation to the environment. From the collection of
\numberofstrategies strategies considered in the Chapter there was not a single
strategy that performed well in that spectrum of noise.

Strategies that have been trained specifically for noisy environments such
as DBS, Evolved FSM 16 Noise 05, Evolved ANN 5 Noise 05, PSO Gambler 2 2 2 Noise 05
and Omega Tit For Tat, performed adequately only in tournaments with restricted
noise. This indicates that possibly there is not a strategy in the literature
trained to be effective for a broad spectrum of noise values. Training such
a strategy would be an interesting avenue of further research. The analysis of the
top performances would then be reproduced whilst including the new trained strategy.

\textbf{Future research - Memory-one strategies}

In Chapter~\ref{chapter:memory_one} the empirical results supported that
extortionate play is not always optimal and that memory-one strategies suffer
from their limited memory in multi agent interactions. All the empirical results
presented have been for the case of two opponents (\(N=2\)). A
future research direction would be to validate the empirical results of the Chapter for
larger values of \(N\).

Another restricted set of strategies on memory that have been studied in the literature
are memory-two strategies. These are strategies that take into account the past
two turns of the match. A compelling research question that arises is whether
the current formulation of Chapter~\ref{chapter:memory_one} can be expanded to
include memory-two strategies, and whether the results still hold.

\textbf{Future research - Training an LSTM strategy}

%TODO based on the conclusions of the Chapter.
