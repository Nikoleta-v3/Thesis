\chapter{Training long short-term memory networks to play the Iterated Prisoner's Dilemma}\label{chapter:lstm}
% TODO another possible title: Machine learning produces successful Iterated Prisoner's Dilemma strategies based on long short-term memory networks

\begin{center}
    The research reported in this Chapter has been carried out with:

    Axerod-Python library version: 4.2.0 \\
    Associated data set: \\ \vspace{.5cm} % TODO archive model weights
\end{center}

\hrulefill

\section{Introduction}

In Chapter~\ref{chapter:meta_tournaments} it was mentioned that consepsualating
and introducing new strategies has been an important aspect of research to the
field. The aim of this Chapter is to introduce new IPD strategies based on an
archetype that has not received much attention in the literature.

In Chapter~\ref{chapter:meta_tournaments} it was concluded that one of the
properties successful strategies in a IPD competition need to have is
cleverness/complexity. Complexity can confer to adaptability, and adaptability
is important in performing well in diverse set of environments. This was
established not only in Chapter~\ref{chapter:meta_tournaments} but also from the
results of Chapter~\ref{chapter:memory_one}. The set of complex strategies that
ranked highly across distinct tournaments in
Chapter~\ref{chapter:meta_tournaments} included strategies based on archetypes
such as finite state automata, hidden Markov models and \textit{artificial
neural networks} (ANNs).

ANNs have successfully been trained to play games exterior to the IPD, such as
checkers~\cite{Chellapilla1999} and chess~\cite{Fogel2004}. The usage of feed
forward ANNs in the IPD was firstly introduced in~\cite{Harrald1996}, and have
been used ever since~\cite{Ashlock2008, Ashlock2006a, Darwen2001, Franken2005}.
Possibly, the most successful ANNs strategies are the one introduced
in~\cite{Harper2017}, as they were ranked \(7^{th}\), \(9^{th}\), and
\(11^{th}\) in a tournament of 223 strategies.

A type of ANNs that have not received much attention in the literature are the
\textit{recurrent neural networks} (RNNs). RNNs are a type of neural networks
that include a feedback connection and are designed to work with inputs in
the form of sequences. RNNs were firstly considered as an archetype in 1996.
In~\cite{Sandholm1996} a RNN which considered a single previous step
as an input was trained via a \(Q\) learning algorithm to play against a single
strategy. The results of~\cite{Sandholm1996} were limited only in the network
learning to play against a single strategy.

The limitations of~\cite{Sandholm1996} could potentially have been due to the
limitations of the RNNs themselves. As it will be discussed later the RNNs
quickly became unable to learn due to the vanishing gradient problem.
To improve on the standard recurrent networks a new network called the
long short-term memory networks (LSTMs) were introduced in~\cite{Hochreiter1997}.
LSTMs do not suffer from the vanishing gradient, can be successfully trained and
are being used in a number of innovative applications such as time series analysis
\cite{Malhotra2015}, speech recognition~\cite{Sak2014} and prediction in
medical care pathways~\cite{Wang2018_lstm}.

LSTMs have not received attention in the IPD literature. This Chapter trains and
introduced a number of strategies  based on LSTMs. The training has been
possible due to the collection of best response sequences generated in
Chapter~\ref{chapter:best_response_sequence}. The performance of the new
strategies is evaluated and compared in a meta tournament analysis of standard
tournaments. The results demonstrate % TODO add results

This Chapter is structured as follows:

\begin{itemize}
    \item section~\ref{section:artificial_neural_networks}, presents an
    introduction to artificial, recurrent and long short-term memory networks.
    \item section~\ref{section:training_a_rnn}, cover the specifics of the LSTMs
    trained in ths Chapter. Their architectural details are presented as well as
    the different training sets they have been trained on.
    \item section~\ref{section:rnn_strategy_validation}, evaluates and compares
    the performance of \lstmstrategies LSTMs based strategies in
    \metatournamentslstm standard IPD computer tournaments.
\end{itemize}

\section{Artificial, recurrent neural and long short-term memory networks}\label{section:artificial_neural_networks}

ANNs are computing systems vaguely inspired by the
biological neural networks that constitute animal brains. ANNs can be viewed as
weighted directed graphs in which artificial neurons are nodes and directed
edges, with weights, are connections between the neuron outputs and the neuron
inputs~\cite{Jain1996}.

The research on ANNs has experienced three periods of extensive activity, as
stated in~\cite{Jain1996}. The first peak was in the 1940s
with~\cite{McCulloch1943} that opened the subject by creating a computational
model for neural networks. The second occurred in 1960s with the Rosenblatt's
convergence theorem of single layer networks~\cite{Rosenblatt1961}, and
the work of~\cite{Minsky1969} which showed their limitations.
The results of~\cite{Minsky1969} halted the enthusiasm of most
researchers in ANNs, resulting in no activity in field for almost 20 years. In
the early 1980s,~\cite{Werbos1974} introduced the back propagation learning
algorithm for multilayer networks. Though the algorithm was
initially proposed by ? it has been reinvented several times and it was
popularizes by~\cite{McClelland1986}. Since then ANNs have received considerable
interest, and today ANNs have been used in a number of innovative applications
such as \cite{Covington2016, Kalogirou2000}.

ANNs based on the connection pattern/architecture can be grouped into two
categories:

\begin{itemize}
    \item \textit{feed forward} networks, in which graphs have no loops, and
    \item recurrent (or feedback) networks, in which loops occur
    because of feedback connections.
\end{itemize}

A graphical representation of a feed forward network with a single hidden layer
is given by Figure~\ref{fig:ann}.

\begin{figure}[!htbp]
    \centering
    \includestandalone[width=.5\textwidth]{src/chapters/07/neural_network}
    \caption{Graphical representation of an artificial neural network (ANN)}.\label{fig:ann}
\end{figure}

A feed forward network takes as input a vector \(x\). Each element of the input
vector is connected to a hidden layer of neurons via a set of learned weights.
The \(j^{\text{th}}\) hidden neuron outputs \(h_j = \phi (\sum_{i} w_{ij}
x_{i})\), where \(\phi\) is an \textit{activation function}. In turn, the hidden
layer is fully connected to an output layer. The \(j^{\text{th}}\) output neuron
outputs \(y_{j} = \sum_{i} v_{ij} h_{i}\). The output layer can be transformed
to probabilities via a softmax function. 

In matrix notation a single layered feed forward network can be described by
the equations,

\begin{align}\label{eq:neural_network_equations}
h & = \phi(Wx) \\ \label{eq:neural_network_equations_two}
y & = Vh
\end{align}

where \(W\) is a weight matrix connecting the input and hidden layers, \(V\) is
a weight matrix connecting the hidden and output layers.

Common activation functions for \(\phi\) include:

\begin{itemize}
    \item the sigmoid function \(\sigma(x)\) which squashes numbers into the range \([0, 1]\)
    \item the hyperbolic tangent, \(tanh(x)\) which squashes numbers into the range \([-1, 1]\)
    \item the rectified linear unit, \(ReLU(x)=max(0,x)\)
\end{itemize}

An extension of feed forward networks are the recurrent networks. RNNs RNNs is
the idea that they might be able to connect previous information to he present
task. They can can be thought of as multiple copies of the same network, each
passing a message to a successor, as demonstrated by Figure~\ref{fig:rnn}.

\begin{figure}[!htbp]
    \centering
    \includestandalone[width=\textwidth]{src/chapters/07/reccurent_neural_network}
    \caption{Graphical representation of a recurrent neural network (RNN)}.\label{fig:rnn}
\end{figure}

RNNs introduce the notion of \textit{internal knowledge}, which can be thought of
as pieces of information that the network maintains over time. The hidden layers
of neural networks encode useful information regarding their inputs, and that
information is pass down by the considering,

\begin{align}\label{eq:recurrent_neural_network_equations}
    h_t & = \phi(Wx_t + Uh_{t-1}) \\
    y_t & = Vh
\end{align}

where \(h_{t-1}\) is the hidden state computed at time \(t-1\) multiplied by
some weight vector \(U\).

Unfortunately, in practice as that \(t\) grows, RNNs become unable to learn to
connect the information. The problem of RNNs and it's fundamental difficulties
were explored in depth by~\cite{Bengio1994}. However, the more time steps we
have, the more chance we have of back-propagation gradients either accumulating
and exploding or vanishing down to nothing.

The LSTM, introduced by~\cite{Hochreiter1997}, a network specifically designed
to avoid the long-term dependency problem. LSTMs work tremendously well on a
large variety of problems, and are now widely used. Remembering information for
long periods of time is practically their default behavior, not something they
struggle to learn.

The way it does so is by creating an internal memory state which is simply added
to the processed input. The time dependence and effects of previous inputs are
controlled by an interesting concept called a forget gate, which determines
which states are remembered or forgotten. Two other gates, the input gate and
output gate, are also featured in LSTM cells.

The core idea to LSTMs is the \textit{cell state} also refereed to as the
long term memory, denote as \(C_t\). The cell state is designed to pass down
information with only a few carefully regulated changes being applied to it
by structures called \textit{gates}. Gates are composed out of a sigmoid neural
net layer and a point wise multiplication operator. To explain the how LSTMs
can the cell gate work consider a LSTM cell at point \(t\) which is given by
Figure~\ref{fig:lstm_cell}.

\begin{figure}[!htbp]
    \centering
    \includestandalone[width=.7\textwidth]{src/chapters/07/lstm_cell}
    \caption{An LSTM cell at point \(t\).}\label{fig:lstm_cell}
\end{figure}

The initial step is to decide what information from the cell state is to be
discarded at the time step. This decision is made by the \textit{forget state}.
The forget state considers the hidden state at time \(t-1\) and the input at time
\(t\). Thus, the forget gate is given by,

\begin{equation}\label{eq:forget_gate}
    f_{t} = \sigma(W_{f}x_{t} + U_{f}h_{t-1} + b_{f})
\end{equation}

where \(b_{f}\) is the input bias.

The next step is to compute the information that is going to be stored at the
cell state from the input \(x_t\). Firstly, the input gate decides which values
are going to be updated. The input gate is given by,

\begin{equation}\label{eq:input_gate}
    i_{t} = \sigma(W_{i}x_{t} + U_{i}h_{t-1} + b_{i}).
\end{equation}

Next, a tanh layer creates a vector of candidate values denoted as \(\tilde{C}_{t}\)
that could be added to the cell state.

\begin{equation}\label{eq:input_gate}
    \tilde{C}_{t} = tanh(W_{c}x_{t} + U_{c}h_{t-1} + b_{c}).
\end{equation}

The cell state \(C_{t}\) using the forget and input gates and the candidate
set of new values. The cell state \(C_{t-1}\) is multiplied by \(f_{t}\), forgetting
the values which have been decided to discard. The new candidate values are
scaled by how much information has been decided to keep from the input. Thus,

\begin{equation}\label{eq:input_gate}
    C_{t} = f_{t} * C_{t-1} + i_{t} * \tilde{C}_{t}.
\end{equation}

The LSTM outputs a hidden state at each step. This is either used to through
an activation function to output \(y_{t}\), and it is passed to the next time
step cell. The output is based on the current cell state. The cell state goes
through a tanh function and it is multiplied by a sigmoid gate that decides which
parts to output from the cell state.

\begin{align}\label{eq:input_gate}
    o_{t} & = \sigma(W_{o}x_{t} + U_{o}h_{t-1} + b_{o}), \\
    h_{t} & = o_{t} * tanh(C_{t})
\end{align}

A representation of an LSTM network is given by Figure~\ref{fig:lstm}.

\begin{figure}[!htbp]
    \centering
    \includestandalone[width=\textwidth]{src/chapters/07/lstm}
    \caption{Graphical representation of an LSTM network.}\label{fig:lstm}
\end{figure}

The LSTM is a complex architecture and contains a large number of parameters
that need to be trained. The LSTM networks have managed to achieve remarkable
results. The training of an LSTM model in this Chapter is carried with the
project. The training data is the data set of best responses generated in
Chapter~\ref{chapter:best_response_sequences} and has been trained using the
facilities of Cardiff's University supercomputer. The details for the training
are presented in section~\ref{section:training_a_rnn}.

\section{Training LSTM networks to play the Iterated Prisoner's Dilemma}\label{section:training_a_rnn}

The LSTMs of this Chapter are trained based on the best response sequence
data set that was generated in Chapter~\ref{chapter:best_response_sequence}.
Consider the strategy Adaptive and a best response to it as presented in
section~\ref{section:the_collection_of_best_response_sequences}.

\begin{table}[htb]
    \centering
    \begin{tabular}{cccccccccccccccc}
        & \textbf{1} & \textbf{2} & \textbf{3} & \textbf{4}  & \textbf{5} & \textbf{6} & \textbf{7} & \textbf{8} & \textbf{9} & \textbf{10} & \textbf{11}  & \textbf{12} &  \(\dots\)  & \textbf{204} &  \textbf{205} \\ 
        \midrule
        Adaptive & \(C\) & \(C\) & \(C\) & \(C\) & \(C\) & \(C\) & \(D\) & \(D\) & \(D\) & \(D\) & \(D\)& \(C\)& \(\dots\) & \(C\) & \(C\) \\
        \(S^*\) & \(C\) & \(C\) & \(D\) & \(D\) & \(D\) & \(D\) & \(D\) & \(D\) & \(D\) & \(D\) & \(D\) & \(D\)& \(\dots\) & \(D\) & \(D\) \\ \bottomrule
    \end{tabular}
    \caption{The interactions between Adaptive and one of the best response sequences
    to the strategy.}\label{table:adaptive_vs_best_response}
    \end{table}

For \(0 \to D\) and \(1 \to C\) Table~\ref{table:adaptive_vs_best_response} is given by,

\begin{table}[htbp]
    \centering
    \begin{tabular}{cccccccccccccccc}
        & \textbf{1} & \textbf{2} & \textbf{3} & \textbf{4}  & \textbf{5} & \textbf{6} & \textbf{7} & \textbf{8} & \textbf{9} & \textbf{10} & \textbf{11} &  \textbf{12} & \(\dots\)  & \textbf{204} &  \textbf{205} \\ 
        \midrule
        Adaptive & 1 & 1 & 1 & 1 & 1 & 1 & 1 & 0 & 0 & 0 & 0& 1& \(\dots\) & 1 & 1 \\
        \(S^*\) & 1 & 1 & 0 & 0 & 0 & 0 & 0 & 0 & 0 & 0 & 0 & 0& \(\dots\) & 0 & 0 \\ \bottomrule
    \end{tabular}
    \caption{The interactions between Adaptive and one of the best response sequences
    to the strategy were \(0 \to D\) and \(1 \to C\).}\label{table:adaptive_vs_best_response_binary}
\end{table}

In order to train the LSTM as a player the network should predict the best
response to strategy's action. Thus, the input to the network is the actions of
the opponents \(Q\), discarding the last turn.

\begin{figure}[!htbp]
    \centering
    \includestandalone[width=\textwidth]{src/chapters/07/input}
\end{figure}

and the expected output it's the response of the \(S^*\) to those actions.
Thus the genes from \(t=2\) instead of \(t=1\).

\begin{figure}[!htbp]
    \centering
    \includestandalone[width=\textwidth]{src/chapters/07/output}
    % \caption{A graphical representation of a sequence to sequence LSTM model}.\label{fig:sequence_to_sequence}
\end{figure}

Two variants of LSTM network have been training in this Chapter. These are refereed to:

\begin{itemize}
    \item sequence to sequence
    \item sequence to probability
\end{itemize}

The sequence to sequence model as an input of actions a each time step by a strategy
an it outputs a response for each at each time step. The sequence to probability
has an input the action of strategy up to time \(t\), and outputs a response
to the history at time \(t\).

A graphical representation of the two networks are given by Figures~\ref{fig:sequence_to_sequence} and
\ref{fig:sequence_to_probability}.

\begin{figure}[!htbp]
    \centering
    \includestandalone[width=\textwidth]{src/chapters/07/sequence_to_sequence}
    \caption{A graphical representation of a sequence to sequence LSTM model}.\label{fig:sequence_to_sequence}
\end{figure}

\begin{figure}[!htbp]
    \centering
    \includestandalone[width=\textwidth]{src/chapters/07/sequence_to_probability}
    \caption{A graphical representation of a sequence to probability LSTM model}.\label{fig:sequence_to_probability}
\end{figure}

The networks are constructed in such way as the length of sequences they receive
it is not fixed. The are not trained for the \totalsequences but each
best response sequence is broken down to 204 sets of input.

\begin{equation}
    \begin{bmatrix}
        1 &  &  \\
        1 & 1 &  \\
        1 & 1 & 1 \\
        1 & 1 & 1 & 1 \\
        1 & 1 & 1 & 1 & 1 \\
        1 & 1 & 1 & 1 & 1 & 1\\
        1 & 1 & 1 & 1 & 1 & 1 & 0 \\
        \vdots & \vdots & \vdots & \vdots & \vdots & \vdots & \vdots \\
    \end{bmatrix} \to 
    \begin{bmatrix}
        1 &  &  \\
        1 & 1 &  \\
        1 & 1 & 0 \\
        1 & 1 & 0 & 0 \\
        1 & 1 & 0 & 0 & 0 \\
        1 & 1 & 0 & 0 & 0 & 0\\
        1 & 1 & 0 & 0 & 0 & 0 & 0 \\
        \vdots & \vdots & \vdots & \vdots & \vdots & \vdots & \vdots \\
    \end{bmatrix}
\end{equation}

\begin{equation}
    \begin{bmatrix}
        1 &  &  \\
        1 & 1 &  \\
        1 & 1 & 1 \\
        1 & 1 & 1 & 1 \\
        1 & 1 & 1 & 1 & 1 \\
        1 & 1 & 1 & 1 & 1 & 1\\
        1 & 1 & 1 & 1 & 1 & 1 & 0 \\
        \vdots & \vdots & \vdots & \vdots & \vdots & \vdots & \vdots \\
    \end{bmatrix} \to 
    \begin{bmatrix}
        1 \\
        1 \\
        0 \\
        0 \\
        0 \\
        0 \\
        0 \\
        \vdots \\
    \end{bmatrix}
\end{equation}

\subsection{Building the networks with Keras}

To construct and train the network model the open-source neural-network package
Keras~\cite{Chollet2015} is used. Its library written in Python, designed to
enable fast experimentation with neural networks.
Neural layers, cost functions, optimizers, initialization schemes, activation
functions, and regularization schemes are all standalone modules that you can
combine to create new models.

The code for implementing the sequence to sequence model is given by
Figure~\ref{fig:keras_sequence_to_sequence}. There are two main types of models
available in Keras: the Sequential model, and the Model class. The Sequential
model is a linear stack of layers. There is a single layer of an LSTM model that
returns the hidden state at each time step. The input shape is (None, 1) because
there is not a fixed length and there is a single time step between the of the
input. There are 100 the outputs turn into a probability with a sigmoid . The
dropout layer 
There are a total of 41301 parameters that need to be trained for the sequence
to sequence model.

\begin{figure}[!htbp]
\begin{usagepy}
>>> from keras.models import Sequential

>>> from keras.layers import (
...     Dense,
...     Dropout,
...     CuDNNLSTM,
... )

>>> num_hidden_cells = 100
>>> drop_out_rate = 0.2

>>> model = Sequential()

>>> model.add(
...    CuDNNLSTM(
...        num_hidden_cells, return_sequences=True, input_shape=(None, 1))
...)

>>> model.add(Dropout(rate=drop_out_rate))

>>> model.add(Dense(1, activation="sigmoid"))
>>> model.summary()
Model: "sequential_1"
_________________________________________________________________
Layer (type)                 Output Shape              Param #   
=================================================================
cu_dnnlstm_1 (CuDNNLSTM)     (None, None, 100)         41200     
_________________________________________________________________
dropout_1 (Dropout)          (None, None, 100)         0         
_________________________________________________________________
dense_1 (Dense)              (None, None, 1)           101       
=================================================================
Total params: 41,301
Trainable params: 41,301
Non-trainable params: 0
_________________________________________________________________

\end{usagepy}
\caption{Keras sequence to sequence.}\label{fig:keras_sequence_to_sequence}
\end{figure}

Similarly, for the sequence to probability network the implementation of the
network using Keras is given by Figure~\ref{fig:keras_sequence_to_probability}.
There are now two LSTM layers. The first one outputs a hidden state at each time
step and the second layer outputs only at time \(t\) the output then goes
through Dropout then turned into probability with a sigmoid function. There are
total of 122101 parameters.

\begin{figure}[!htbp]
\begin{usagepy}
    >>> from keras.models import Sequential
>>> from keras.layers import (
...     Dense,
...     Dropout,
...     CuDNNLSTM,
... )

>>> num_hidden_cells = 100
>>> drop_out_rate = 0.2

>>> model = Sequential()

>>> model.add(
...     CuDNNLSTM(num_hidden_cells, return_sequences=True, input_shape=(None, 1))
... )

>>> model.add(CuDNNLSTM(num_hidden_cells))
>>> model.add(Dropout(rate=drop_out_rate))

>>> model.add((Dense(1, activation="sigmoid")))
>>> model.summary()
Model: "sequential_2"
_________________________________________________________________
Layer (type)                 Output Shape              Param #   
=================================================================
cu_dnnlstm_2 (CuDNNLSTM)     (None, None, 100)         41200     
_________________________________________________________________
cu_dnnlstm_3 (CuDNNLSTM)     (None, 100)               80800     
_________________________________________________________________
dropout_2 (Dropout)          (None, 100)               0         
_________________________________________________________________
dense_2 (Dense)              (None, 1)                 101       
=================================================================
Total params: 122,101
Trainable params: 122,101
Non-trainable params: 0
_________________________________________________________________

\end{usagepy}
\caption{Keras sequence to probability.}\label{fig:keras_sequence_to_probability}
\end{figure}

Keras contains two implementations on an LSTM model. That is the class LSTM and
the CCNLSTM which was used here. CCNLSTM is a fast  LSTM implementation with
NVIDIA CUDA Deep Neural Network library (cuDNN) which run on a graphics
processing unit (GPU).

\subsection{Training on GPU}

Conventionally most training and running computer code happens on the central
processing unit (CPU). A CPU also called a central processor or main processor
is essentially the brain of any computing device, carrying out the instructions
of a program by performing control, logical, and input/output (I/O) operations.
CPUs, can have a number of cores, and its basic design and purpose—to process
complex computations—has not really changed. A CPU core is designed to support
an extremely broad variety of tasks.

A graphical processing unit (GPU), on the other hand, has smaller-sized but many
more logical cores (arithmetic logic units or ALUs, control units and memory
cache) whose basic design is to process a set of simpler and more identical
computations in parallel.

Initially designed mainly as dedicated graphical rendering workhorses of
computer games, GPUs were later enhanced to accelerate other geometric
calculations (for instance, transforming polygons or rotating verticals into
different coordinate systems like 3D). Nvidia created a parallel computing
architecture and platform for its GPUs called CUDA, which gave developers access
and the ability to express simple processing operations in parallel through
code.

While a CPU core is more powerful than a GPU core, the vast majority of this
power goes unused by ML applications. whereas a GPU core is optimized
exclusively for data computations. Because of this singular focus, a GPU core is
simpler and has a smaller die area than a CPU, allowing many more GPU cores to
be crammed onto a single chip. Consequently, ML applications, which perform
large numbers of computations on a vast amount of data, can see huge (i.e., 5 to
10 times) performance improvements when running on a GPU versus a CPU..
As many have said GPUs are so fast because they are so efficient for matrix
multiplication.

Keras CCNLSTM class makes running on GPU easy. Cardiff's super computer
was used for the training of the models.

\subsection{Training data sets}

It was explained in the the best response data set from Chapter 6 is used as
training data. Moreover, each best responses has broken down to 204 different
inputs. In order to understand the difference the effect of the training data
the models have been trained on different subsets of the data set. More
specifically in total of four different sub sets. These with a brief explanation
and number are given by Table~\ref{table:training_data_sets}.

\newcolumntype{g}{>{\columncolor{Gray}}c}
\begin{table}[htb]
    \centering
    \resizebox{\textwidth}{!}{
    \begin{tabular}{cglg}
    \toprule
    Data set & number of \(Q\) & Explanation & number of \(S^*\) \\
    \midrule
    all strategies & \numberofstrategiesbestsequences & The data set as presented in Chapter~\ref{chapter:best_response_sequence} & \totalsequences \\
    top performing strategies & \topstrategies & Top \topstrategies performing strategies in a standard tournaments performed with APL & \topsequences \\
    strategies across ranks & \acrossstrategies & A set \acrossstrategies that there rank is across & \acrosssequences\\
    basic strategies & \basicstrategies & A set \basicstrategies which are classified as the basic in APL & \basicsequences \\ 
    \bottomrule
    \end{tabular}}
    \caption{Subset of data set that the networks are trained on. The APL project
    performs a round robin tournament with all the strategies implemented on every release.
    The top strategies and the strategies across are based on the results of the tournament
    with APL version 4.5.0.}\label{table:training_data_sets}
\end{table}

Due to the different size of the data set the networks have trained for a different
number of epochs. The number of epochs is the number of complete passes through
the training data set. The number of epochs each network has trained for each
data set is given by Table~\ref{table:epochs}.

\begin{table}[!htbp]
    \begin{center}
    \resizebox{.9\textwidth}{!}{
        \begin{tabular}{lrrrr}
\toprule
{} &  all strategies &  top strategies &  strategies across ranks &  basic strategies \\
\midrule
sequence to sequence    &           19999 &           83999 &                    39999 &            129999 \\
sequence to probability &            7999 &           74999 &                    39999 &             84999 \\
\bottomrule
\end{tabular}

    }
\end{center}
\caption{Number of epochs each both networks, sequence to sequence and 
sequence to probability, were trained for each subset.}\label{table:epochs}
\end{table}

The data are split into training and test training set. The test set is
20\% of the data. The validation and loss is calculated at each epoch.
The validation is the measure that. The loss is based on the.


\begin{figure}[!htbp]
    \begin{subfigure}{\textwidth}
    \centering
    \includegraphics[width=.8\textwidth]{src/chapters/07/img/validation_plot_all_strategies.pdf}
    \end{subfigure}\hfill
    \begin{subfigure}{\textwidth}
    \centering
    \includegraphics[width=.8\textwidth]{src/chapters/07/img/validation_plot_top_strategies.pdf}
    \end{subfigure}
    \begin{subfigure}{\textwidth}
    \centering
    \includegraphics[width=.8\textwidth]{src/chapters/07/img/validation_plot_across_ranks_strategies.pdf}
    \end{subfigure}
    \begin{subfigure}{\textwidth}
    \centering
    \includegraphics[width=.8\textwidth]{src/chapters/07/img/validation_plot_basic_strategies.pdf}
    \end{subfigure}
    \caption{Validation of sequence to sequence}\label{fig:validation_sequence_to_sequence}
\end{figure}

\begin{figure}[!htbp]
    \begin{subfigure}{\textwidth}
    \centering
    \includegraphics[width=.8\textwidth]{src/chapters/07/img/validation_plot_classification_all_strategies.pdf}
    \end{subfigure}\hfill
    \begin{subfigure}{\textwidth}
    \centering
    \includegraphics[width=.8\textwidth]{src/chapters/07/img/validation_plot_classification_top_strategies.pdf}
    \end{subfigure}
    \begin{subfigure}{\textwidth}
    \centering
    \includegraphics[width=.8\textwidth]{src/chapters/07/img/validation_plot_classification_across_ranks_strategies.pdf}
    \end{subfigure}
    \begin{subfigure}{\textwidth}
    \centering
    \includegraphics[width=.8\textwidth]{src/chapters/07/img/validation_plot_classification_basic_strategies.pdf}
    \end{subfigure}
    \caption{Validation of sequence to probability}\label{fig:validation_sequence_to_probability}
\end{figure}

Discussion on validation. In section~\ref{section:rnn_strategy_validation} the
LSTM networks are validated by as an IPD strategy using a meta tournament
analysis.

\section{Validation of LSTM base strategies using a meta tournament analysis}\label{section:rnn_strategy_validation}

To evaluate the LSTM players as strategies, a strategy class was implemented
that takes as an argument the a Keras model. The player then uses the history
of the match translate it to an input in the form that the networks can then
translate into an action. Note that the network was not trained to consider
it's opening move. The opening move is also an input to the strategy class.
In the meta tournament analysis different strategies that have a different
opening play are considered.

\begin{figure}[!htbp]
\begin{sourcepy}
import numpy as np

import axelrod as axl
from axelrod.random_ import random_choice
from keras.layers import LSTM, Dense, Dropout
from keras.models import Sequential

C, D = axl.Action.C, axl.Action.D


class LSTMPlayer(axl.Player):
    name = "The LSTM player"
    classifier = {
        "memory_depth": float("inf"),
        "stochastic": True,
        "inspects_source": False,
        "manipulates_source": False,
        "manipulates_state": False,
    }

    def __init__(self, model, reshape_history_funct, opening_probability=0.78):
        self.model = model
        self.opening_probability = opening_probability
        self.reshape_history_function = reshape_history_funct
        super().__init__()
        if opening_probability in [0, 1]:
            self.classifier["stochastic"] = False

    def strategy(self, opponent):
        if len(self.history) == 0:
            return random_choice(self.opening_probability)

        history = [action.value for action in opponent.history]
        prediction = float(
            self.model.predict(self.reshape_history_function(history))[0][-1]
        )

        return axl.Action(round(prediction))

    def __repr__(self):
        return self.name
\end{sourcepy}
\end{figure}

The implemented class can interact with any opponent from APL.
An example of usage of the LSTMPlayer is given by Figure.

To evaluate the performance of the LSTM strategy a meta tournament analysis,
similar to Chapter~\ref{chapter:meta_tournaments}, is carried out. The
tournament type considered here are standard tournaments. The data collections
is given by Algorithm~\ref{meta_tournament_lstm_validation}. At each trial
a number of opponents, a list of opponent From 5 to 10. The number of turns and repetitions
are fixed to 200 and 50.

The number 205 was hard coded. The evaluation does not want to take into account
the last turns. 205 was selected so the last turns could have beed discarded
and the match length would be 200 which is the number of turns commonly used
in the literature.

%TODO talk about opening probability

\begin{algorithm}[!htbp]
    \setstretch{1.35}
    \ForEach{\text{seed} $\in [0, 300]$}{
        $N \gets \text{randomly select integer}\in [N_{min}, N_{max}]$\;
        $\text{players} \gets  \text{randomly select $N$ players}$\;
        $\text{players} \gets  \text{players} + \text{LSTM strategy}$\;
        $N \gets N + 1$\;
        $k \gets  50$\;
        $n \gets  200$\;
        \vspace{0.4cm}
        $\text{result standard}$ $\gets$ Axelrod.tournament$(\text{players}, n, k)$\;}
    \KwRet{result standard}\;
    \caption{Data collection Algorithm}
    \label{algorithm:meta_tournament_lstm_validation}
\end{algorithm}

A total of \metatournamentslstm trials of
Algorithm~\ref{algorithm:meta_tournament_lstm_validation} have been performed.
Similarly to Chapter~\ref{chapter:meta_tournament} each trial outputs a result
summary of the tournament. The performance of the strategies are evaluated on
the normalised rank \(r\), and more specifically on the median normalised rank
\(\bar{r}\). A total of different strategies competed in the tournaments, and
the strategy with the maximum participation, excluding the LSTM strategy, was.

There are two distinct networks that have been trained on four different data
sets. Moreover, each networks translates into a player that the opening
probability can have three different values. Thus a total of $2 \times 4 \times
3$ LSTM players performance is evaluated in this section.

\begin{table}[!htbp]
    \begin{center}
    \resizebox{.9\textwidth}{!}{
        \begin{tabular}{lllllll}
\toprule
{} & \multicolumn{3}{l}{sequence to sequence} & \multicolumn{3}{l}{sequence to probability} \\
{} &                         $y=0$ &                         $y=1$ &                      $y=0.78$ &                         $y=0$ &                         $y=1$ &                      $y=0.78$ \\
\midrule
All strategies          &  \cellcolor{orange!67.0}0.667 &  \cellcolor{orange!22.0}0.222 &  \cellcolor{orange!33.0}0.333 &  \cellcolor{orange!78.0}0.778 &  \cellcolor{orange!33.0}0.333 &    \cellcolor{orange!50.0}0.5 \\
Top strategies          &  \cellcolor{orange!71.0}0.714 &  \cellcolor{orange!44.0}0.444 &    \cellcolor{orange!50.0}0.5 &    \cellcolor{orange!50.0}0.5 &  \cellcolor{orange!43.0}0.429 &  \cellcolor{orange!43.0}0.429 \\
Across ranks strategies &   \cellcolor{orange!75.0}0.75 &  \cellcolor{orange!67.0}0.667 &  \cellcolor{orange!68.0}0.683 &    \cellcolor{orange!50.0}0.5 &   \cellcolor{orange!25.0}0.25 &    \cellcolor{orange!30.0}0.3 \\
Basic strategies        &    \cellcolor{orange!80.0}0.8 &    \cellcolor{orange!60.0}0.6 &  \cellcolor{orange!62.0}0.625 &    \cellcolor{orange!80.0}0.8 &    \cellcolor{orange!30.0}0.3 &  \cellcolor{orange!43.0}0.429 \\
\bottomrule
\end{tabular}

    }
\end{center}
\caption{Median normalised ranks of th 24 LSTM strategies over standard tournaments.}
% \label{table:normalised_rank_lstm_tournaments}
\end{table}

\begin{figure}[!htbp]
    \begin{subfigure}{\textwidth}
    \includegraphics[width=\textwidth]{src/chapters/07/img/normalised_rank_all_strategies.pdf}
    \end{subfigure}
    \par\bigskip
    \begin{subfigure}{\textwidth}
    \includegraphics[width=\textwidth]{src/chapters/07/img/normalised_rank_top_strategies.pdf}
    \end{subfigure}
    \par\bigskip
    \begin{subfigure}{\textwidth}
    \includegraphics[width=\textwidth]{src/chapters/07/img/normalised_rank_across_ranks_strategies.pdf}
    \end{subfigure}
    \par\bigskip
    \begin{subfigure}{\textwidth}
    \includegraphics[width=\textwidth]{src/chapters/07/img/normalised_rank_basic_strategies.pdf}
    \end{subfigure}
    % \caption{Validation of sequence to sequence}\label{fig:validation_sequence_to_sequence}
\end{figure}


\begin{table}[!htbp]
    \begin{center}
    \resizebox{.9\textwidth}{!}{
        \begin{tabular}{llrrrrrrrrrrrr}
\toprule
                 &            &  count &   mean &    std &  min &    10\% &    25\% &    50\% &    75\% &    95\% &  max &   skew &   kurt \\
\midrule
\rowcolor{Gray}
All strategies & $p_o=0$ &  300.0 &  0.620 &  0.278 &  0.0 &  0.200 &  0.429 &  0.667 &  0.839 &  1.000 &  1.0 & -0.523 & -0.587 \\
\rowcolor{Gray}
                 & $p_o=1$ &  300.0 &  0.295 &  0.252 &  0.0 &  0.000 &  0.111 &  0.222 &  0.458 &  0.779 &  1.0 &  0.702 & -0.251 \\
\rowcolor{Gray}
                 & $p_o=0.78$ &  300.0 &  0.368 &  0.265 &  0.0 &  0.000 &  0.143 &  0.333 &  0.560 &  0.833 &  1.0 &  0.433 & -0.647 \\
                 \midrule
Top strategies & $p_o=0$ &  300.0 &  0.655 &  0.273 &  0.0 &  0.250 &  0.500 &  0.714 &  0.875 &  1.000 &  1.0 & -0.629 & -0.436 \\
                 & $p_o=1$ &  300.0 &  0.461 &  0.274 &  0.0 &  0.090 &  0.222 &  0.444 &  0.667 &  0.875 &  1.0 & -0.029 & -0.940 \\
                 & $p_o=0.78$ &  300.0 &  0.509 &  0.255 &  0.0 &  0.167 &  0.333 &  0.500 &  0.704 &  0.876 &  1.0 & -0.158 & -0.710 \\
                 \midrule
\rowcolor{Gray}
Across ranks strategies & $p_o=0$ &  300.0 &  0.643 &  0.306 &  0.0 &  0.141 &  0.486 &  0.750 &  0.875 &  1.000 &  1.0 & -0.768 & -0.523 \\
\rowcolor{Gray}
                 & $p_o=1$ &  300.0 &  0.636 &  0.262 &  0.0 &  0.250 &  0.500 &  0.667 &  0.833 &  1.000 &  1.0 & -0.680 & -0.198 \\
\rowcolor{Gray}
                 & $p_o=0.78$ &  300.0 &  0.645 &  0.255 &  0.0 &  0.250 &  0.500 &  0.683 &  0.833 &  1.000 &  1.0 & -0.754 & -0.034 \\
                 \midrule
Basic strategies & $p_o=0$ &  300.0 &  0.728 &  0.263 &  0.0 &  0.333 &  0.600 &  0.800 &  1.000 &  1.000 &  1.0 & -0.949 &  0.229 \\
                 & $p_o=1$ &  300.0 &  0.556 &  0.283 &  0.0 &  0.143 &  0.375 &  0.600 &  0.778 &  1.000 &  1.0 & -0.365 & -0.803 \\
                 & $p_o=0.78$ &  300.0 &  0.579 &  0.280 &  0.0 &  0.167 &  0.375 &  0.625 &  0.800 &  1.000 &  1.0 & -0.435 & -0.772 \\
\bottomrule
\end{tabular}

    }
\end{center}
% \caption{Number of epochs each both networks, sequence to sequence and 
% sequence to probability, were trained for each subset.}\label{table:epochs}
\end{table}


\begin{figure}[!htbp]
    \begin{subfigure}{\textwidth}
    \includegraphics[width=\textwidth]{src/chapters/07/img/normalised_rank_classification_all_strategies.pdf}
    \end{subfigure}
    \par\bigskip
    \begin{subfigure}{\textwidth}
    \includegraphics[width=\textwidth]{src/chapters/07/img/normalised_rank_classification_top_strategies.pdf}
    \end{subfigure}
    \par\bigskip
    \begin{subfigure}{\textwidth}
    \includegraphics[width=\textwidth]{src/chapters/07/img/normalised_rank_classification_across_ranks_strategies.pdf}
    \end{subfigure}
    \par\bigskip
    \begin{subfigure}{\textwidth}
    \includegraphics[width=\textwidth]{src/chapters/07/img/normalised_rank_classification_basic_strategies.pdf}
    \end{subfigure}
    % \caption{Validation of sequence to probability}\label{fig:validation_sequence_to_probability}
\end{figure}

\begin{table}[!htbp]
    \begin{center}
    \resizebox{.9\textwidth}{!}{
        \begin{tabular}{llrrrrrrrrrr}
\toprule
                 &          &  count &   mean &    std &  min &    10\% &    25\% &    50\% &    75\% &    95\% &  max \\
\midrule
All strategies & $y=0$ &  300.0 &  0.720 &  0.254 &  0.0 &  0.333 &  0.571 &  0.778 &  0.900 &  1.000 &  1.0 \\
                 & $y=1$ &  300.0 &  0.339 &  0.244 &  0.0 &  0.000 &  0.125 &  0.333 &  0.500 &  0.800 &  1.0 \\
                 & $y=0.78$ &  300.0 &  0.471 &  0.255 &  0.0 &  0.125 &  0.286 &  0.500 &  0.625 &  0.875 &  1.0 \\
Top strategies & $y=0$ &  300.0 &  0.491 &  0.305 &  0.0 &  0.000 &  0.200 &  0.500 &  0.714 &  1.000 &  1.0 \\
                 & $y=1$ &  300.0 &  0.417 &  0.283 &  0.0 &  0.000 &  0.167 &  0.429 &  0.600 &  0.875 &  1.0 \\
                 & $y=0.78$ &  300.0 &  0.432 &  0.277 &  0.0 &  0.000 &  0.200 &  0.429 &  0.625 &  0.876 &  1.0 \\
Across ranks strategies & $y=0$ &  300.0 &  0.487 &  0.287 &  0.0 &  0.000 &  0.286 &  0.500 &  0.700 &  1.000 &  1.0 \\
                 & $y=1$ &  300.0 &  0.308 &  0.267 &  0.0 &  0.000 &  0.111 &  0.250 &  0.500 &  0.800 &  1.0 \\
                 & $y=0.78$ &  300.0 &  0.335 &  0.272 &  0.0 &  0.000 &  0.125 &  0.300 &  0.556 &  0.800 &  1.0 \\
Basic strategies & $y=0$ &  290.0 &  0.738 &  0.239 &  0.0 &  0.400 &  0.600 &  0.800 &  0.975 &  1.000 &  1.0 \\
                 & $y=1$ &  290.0 &  0.323 &  0.249 &  0.0 &  0.000 &  0.125 &  0.300 &  0.500 &  0.778 &  1.0 \\
                 & $y=0.78$ &  290.0 &  0.432 &  0.269 &  0.0 &  0.000 &  0.200 &  0.429 &  0.625 &  0.875 &  1.0 \\
\bottomrule
\end{tabular}

    }
\end{center}
% \caption{Number of epochs each both networks, sequence to sequence and 
% sequence to probability, were trained for each subset.}\label{table:epochs}
\end{table}

\begin{figure}[!htbp]
    \begin{subfigure}{.45\textwidth}
    \includegraphics[width=\textwidth]{src/chapters/07/img/cfd_to_sequence_all_strategies.pdf}
    \caption{\(P(r<0.5): 0.350 P(r<0.5): 0.797 P(r<0.5): 0.717\)}
    \end{subfigure}\hfill
    \begin{subfigure}{.45\textwidth}
    \includegraphics[width=\textwidth]{src/chapters/07/img/cfd_to_sequence_top_strategies.pdf}
    \caption{\(P(r<0.5): 0.310 P(r<0.5): 0.573 P(r<0.5): 0.523\)}
    \end{subfigure}
    \begin{subfigure}{.45\textwidth}
    \includegraphics[width=\textwidth]{src/chapters/07/img/cfd_to_sequence_across_ranks_strategies.pdf}
    \caption(\(P(r<0.5): 0.313 P(r<0.5): 0.303 P(r<0.5): 0.280\))
    \end{subfigure}\hfill
    \begin{subfigure}{.45\textwidth}
    \includegraphics[width=\textwidth]{src/chapters/07/img/cfd_to_sequence_basic_strategies.pdf}
    \caption{\(P(r<0.5): 0.207 P(r<0.5): 0.420 P(r<0.5): 0.387\)}
    \end{subfigure}
    % \caption{Validation of sequence to probability}\label{fig:validation_sequence_to_probability}
\end{figure}


\begin{figure}[!htbp]
    \begin{subfigure}{.45\textwidth}
    \includegraphics[width=\textwidth]{src/chapters/07/img/cfd_to_probability_all_strategies.pdf}
    \caption{\(P(r<0.5): 0.207 P(r<0.5): 0.783 P(r<0.5): 0.567\)}
    \end{subfigure}\hfill
    \begin{subfigure}{.45\textwidth}
    \includegraphics[width=\textwidth]{src/chapters/07/img/cfd_to_probability_top_strategies.pdf}
    \caption{\(P(r<0.5): 0.503 P(r<0.5): 0.633 P(r<0.5): 0.610\)}
    \end{subfigure}
    \begin{subfigure}{.45\textwidth}
    \includegraphics[width=\textwidth]{src/chapters/07/img/cfd_to_probability_across_ranks_strategies.pdf}
    \caption(\(P(r<0.5): 0.550 P(r<0.5): 0.763 P(r<0.5): 0.733\))
    \end{subfigure}\hfill
    \begin{subfigure}{.45\textwidth}
    \includegraphics[width=\textwidth]{src/chapters/07/img/cfd_to_probability_basic_strategies.pdf}
    \caption{\(P(r<0.5): 0.207, P(r<0.5): 0.779, P(r<0.5): 0.607\)}
    \end{subfigure}
    % \caption{Validation of sequence to probability}\label{fig:validation_sequence_to_probability}
\end{figure}

\subsection{Fingerprinting the LSTM based strategies}

\newpage

\begin{figure}[!htbp]
    \begin{subfigure}{\textwidth}
        \includegraphics[width=.3\textwidth]{"src/chapters/07/img/Tit For Tat_lstm_sequence_0".pdf}
        \includegraphics[width=.3\textwidth]{"src/chapters/07/img/Tit For Tat_lstm_sequence_1".pdf}
        \includegraphics[width=.3\textwidth]{"src/chapters/07/img/Tit For Tat_lstm_sequence_0_78".pdf}
    \end{subfigure}
    \begin{subfigure}{\textwidth}
        \includegraphics[width=.3\textwidth]{"src/chapters/07/img/Tit For Tat_top_twenty_sequence_0".pdf}
        \includegraphics[width=.3\textwidth]{"src/chapters/07/img/Tit For Tat_top_twenty_sequence_1".pdf}
        \includegraphics[width=.3\textwidth]{"src/chapters/07/img/Tit For Tat_top_twenty_sequence_0_78".pdf}
    \end{subfigure}
    \begin{subfigure}{\textwidth}
        \includegraphics[width=.3\textwidth]{"src/chapters/07/img/Tit For Tat_twenty_sequence_0".pdf}
        \includegraphics[width=.3\textwidth]{"src/chapters/07/img/Tit For Tat_twenty_sequence_1".pdf}
        \includegraphics[width=.3\textwidth]{"src/chapters/07/img/Tit For Tat_twenty_sequence_0_78".pdf}
    \end{subfigure}
    \begin{subfigure}{\textwidth}
        \includegraphics[width=.3\textwidth]{"src/chapters/07/img/Tit For Tat_basic_sequence_0".pdf}
        \includegraphics[width=.3\textwidth]{"src/chapters/07/img/Tit For Tat_basic_sequence_1".pdf}
        \includegraphics[width=.3\textwidth]{"src/chapters/07/img/Tit For Tat_basic_sequence_0_78".pdf}
    \end{subfigure}
\end{figure}

\begin{figure}[!htbp]
    \begin{subfigure}{\textwidth}
        \includegraphics[width=.3\textwidth]{"src/chapters/07/img/Tit For Tat_lstm_classification_0".pdf}
        \includegraphics[width=.3\textwidth]{"src/chapters/07/img/Tit For Tat_lstm_classification_1".pdf}
        \includegraphics[width=.3\textwidth]{"src/chapters/07/img/Tit For Tat_lstm_classification_0_78".pdf}
    \end{subfigure}
    \begin{subfigure}{\textwidth}
        \includegraphics[width=.3\textwidth]{"src/chapters/07/img/Tit For Tat_top_twenty_classification_0".pdf}
        \includegraphics[width=.3\textwidth]{"src/chapters/07/img/Tit For Tat_top_twenty_classification_1".pdf}
        \includegraphics[width=.3\textwidth]{"src/chapters/07/img/Tit For Tat_top_twenty_classification_0_78".pdf}
    \end{subfigure}
    \begin{subfigure}{\textwidth}
        \includegraphics[width=.3\textwidth]{"src/chapters/07/img/Tit For Tat_twenty_classification_0".pdf}
        \includegraphics[width=.3\textwidth]{"src/chapters/07/img/Tit For Tat_twenty_classification_1".pdf}
        \includegraphics[width=.3\textwidth]{"src/chapters/07/img/Tit For Tat_twenty_classification_0_78".pdf}
    \end{subfigure}
    \begin{subfigure}{\textwidth}
        \includegraphics[width=.3\textwidth]{"src/chapters/07/img/Tit For Tat_basic_classification_0".pdf}
        \includegraphics[width=.3\textwidth]{"src/chapters/07/img/Tit For Tat_basic_classification_1".pdf}
        \includegraphics[width=.3\textwidth]{"src/chapters/07/img/Tit For Tat_basic_classification_0_78".pdf}
    \end{subfigure}
\end{figure}

\newpage


\begin{figure}[!htbp]
    \begin{subfigure}{\textwidth}
        \includegraphics[width=.3\textwidth]{"src/chapters/07/img/Win-Shift Lose-Stay_lstm_sequence_0".pdf}
        \includegraphics[width=.3\textwidth]{"src/chapters/07/img/Win-Shift Lose-Stay_lstm_sequence_1".pdf}
        \includegraphics[width=.3\textwidth]{"src/chapters/07/img/Win-Shift Lose-Stay_lstm_sequence_0_78".pdf}
    \end{subfigure}
    \begin{subfigure}{\textwidth}
        \includegraphics[width=.3\textwidth]{"src/chapters/07/img/Win-Shift Lose-Stay_top_twenty_sequence_0".pdf}
        \includegraphics[width=.3\textwidth]{"src/chapters/07/img/Win-Shift Lose-Stay_top_twenty_sequence_1".pdf}
        \includegraphics[width=.3\textwidth]{"src/chapters/07/img/Win-Shift Lose-Stay_top_twenty_sequence_0_78".pdf}
    \end{subfigure}
    \begin{subfigure}{\textwidth}
        \includegraphics[width=.3\textwidth]{"src/chapters/07/img/Win-Shift Lose-Stay_twenty_sequence_0".pdf}
        \includegraphics[width=.3\textwidth]{"src/chapters/07/img/Win-Shift Lose-Stay_twenty_sequence_1".pdf}
        \includegraphics[width=.3\textwidth]{"src/chapters/07/img/Win-Shift Lose-Stay_twenty_sequence_0_78".pdf}
    \end{subfigure}
    \begin{subfigure}{\textwidth}
        \includegraphics[width=.3\textwidth]{"src/chapters/07/img/Win-Shift Lose-Stay_basic_sequence_0".pdf}
        \includegraphics[width=.3\textwidth]{"src/chapters/07/img/Win-Shift Lose-Stay_basic_sequence_1".pdf}
        \includegraphics[width=.3\textwidth]{"src/chapters/07/img/Win-Shift Lose-Stay_basic_sequence_0_78".pdf}
    \end{subfigure}
\end{figure}

\begin{figure}[!htbp]
    \begin{subfigure}{\textwidth}
        \includegraphics[width=.3\textwidth]{"src/chapters/07/img/Win-Shift Lose-Stay_lstm_classification_0".pdf}
        \includegraphics[width=.3\textwidth]{"src/chapters/07/img/Win-Shift Lose-Stay_lstm_classification_1".pdf}
        \includegraphics[width=.3\textwidth]{"src/chapters/07/img/Win-Shift Lose-Stay_lstm_classification_0_78".pdf}
    \end{subfigure}
    \begin{subfigure}{\textwidth}
        \includegraphics[width=.3\textwidth]{"src/chapters/07/img/Win-Shift Lose-Stay_top_twenty_classification_0".pdf}
        \includegraphics[width=.3\textwidth]{"src/chapters/07/img/Win-Shift Lose-Stay_top_twenty_classification_1".pdf}
        \includegraphics[width=.3\textwidth]{"src/chapters/07/img/Win-Shift Lose-Stay_top_twenty_classification_0_78".pdf}
    \end{subfigure}
    \begin{subfigure}{\textwidth}
        \includegraphics[width=.3\textwidth]{"src/chapters/07/img/Win-Shift Lose-Stay_twenty_classification_0".pdf}
        \includegraphics[width=.3\textwidth]{"src/chapters/07/img/Win-Shift Lose-Stay_twenty_classification_1".pdf}
        \includegraphics[width=.3\textwidth]{"src/chapters/07/img/Win-Shift Lose-Stay_twenty_classification_0_78".pdf}
    \end{subfigure}
    \begin{subfigure}{\textwidth}
        \includegraphics[width=.3\textwidth]{"src/chapters/07/img/Win-Shift Lose-Stay_basic_classification_0".pdf}
        \includegraphics[width=.3\textwidth]{"src/chapters/07/img/Win-Shift Lose-Stay_basic_classification_1".pdf}
        \includegraphics[width=.3\textwidth]{"src/chapters/07/img/Win-Shift Lose-Stay_basic_classification_0_78".pdf}
    \end{subfigure}
\end{figure}

\begin{figure}[!htbp]
    \begin{subfigure}{\textwidth}
        \includegraphics[width=.3\textwidth]{"src/chapters/07/img/Stewart_lstm_sequence_0".pdf}
        \includegraphics[width=.3\textwidth]{"src/chapters/07/img/Stewart_lstm_sequence_1".pdf}
        \includegraphics[width=.3\textwidth]{"src/chapters/07/img/Stewart_lstm_sequence_0_78".pdf}
    \end{subfigure}
    \begin{subfigure}{\textwidth}
        \includegraphics[width=.3\textwidth]{"src/chapters/07/img/Stewart_top_twenty_sequence_0".pdf}
        \includegraphics[width=.3\textwidth]{"src/chapters/07/img/Stewart_top_twenty_sequence_1".pdf}
        \includegraphics[width=.3\textwidth]{"src/chapters/07/img/Stewart_top_twenty_sequence_0_78".pdf}
    \end{subfigure}
    \begin{subfigure}{\textwidth}
        \includegraphics[width=.3\textwidth]{"src/chapters/07/img/Stewart_twenty_sequence_0".pdf}
        \includegraphics[width=.3\textwidth]{"src/chapters/07/img/Stewart_twenty_sequence_1".pdf}
        \includegraphics[width=.3\textwidth]{"src/chapters/07/img/Stewart_twenty_sequence_0_78".pdf}
    \end{subfigure}
    \begin{subfigure}{\textwidth}
        \includegraphics[width=.3\textwidth]{"src/chapters/07/img/Stewart_basic_sequence_0".pdf}
        \includegraphics[width=.3\textwidth]{"src/chapters/07/img/Stewart_basic_sequence_1".pdf}
        \includegraphics[width=.3\textwidth]{"src/chapters/07/img/Stewart_basic_sequence_0_78".pdf}
    \end{subfigure}
\end{figure}

\begin{figure}[!htbp]
    \begin{subfigure}{\textwidth}
        \includegraphics[width=.3\textwidth]{"src/chapters/07/img/Stewart_lstm_classification_0".pdf}
        \includegraphics[width=.3\textwidth]{"src/chapters/07/img/Stewart_lstm_classification_1".pdf}
        \includegraphics[width=.3\textwidth]{"src/chapters/07/img/Stewart_lstm_classification_0_78".pdf}
    \end{subfigure}
    \begin{subfigure}{\textwidth}
        \includegraphics[width=.3\textwidth]{"src/chapters/07/img/Stewart_top_twenty_classification_0".pdf}
        \includegraphics[width=.3\textwidth]{"src/chapters/07/img/Stewart_top_twenty_classification_1".pdf}
        \includegraphics[width=.3\textwidth]{"src/chapters/07/img/Stewart_top_twenty_classification_0_78".pdf}
    \end{subfigure}
    \begin{subfigure}{\textwidth}
        \includegraphics[width=.3\textwidth]{"src/chapters/07/img/Stewart_twenty_classification_0".pdf}
        \includegraphics[width=.3\textwidth]{"src/chapters/07/img/Stewart_twenty_classification_1".pdf}
        \includegraphics[width=.3\textwidth]{"src/chapters/07/img/Stewart_twenty_classification_0_78".pdf}
    \end{subfigure}
    \begin{subfigure}{\textwidth}
        \includegraphics[width=.3\textwidth]{"src/chapters/07/img/Stewart_basic_classification_0".pdf}
        \includegraphics[width=.3\textwidth]{"src/chapters/07/img/Stewart_basic_classification_1".pdf}
        \includegraphics[width=.3\textwidth]{"src/chapters/07/img/Stewart_basic_classification_0_78".pdf}
    \end{subfigure}
\end{figure}

\begin{figure}[!htbp]
    \begin{subfigure}{\textwidth}
        \includegraphics[width=.3\textwidth]{"src/chapters/07/img/Beautil_lstm_sequence_0".pdf}
        \includegraphics[width=.3\textwidth]{"src/chapters/07/img/Beautil_lstm_sequence_1".pdf}
        \includegraphics[width=.3\textwidth]{"src/chapters/07/img/Beautil_lstm_sequence_0_78".pdf}
    \end{subfigure}
    \begin{subfigure}{\textwidth}
        \includegraphics[width=.3\textwidth]{"src/chapters/07/img/Beautil_top_twenty_sequence_0".pdf}
        \includegraphics[width=.3\textwidth]{"src/chapters/07/img/Beautil_top_twenty_sequence_1".pdf}
        \includegraphics[width=.3\textwidth]{"src/chapters/07/img/Beautil_top_twenty_sequence_0_78".pdf}
    \end{subfigure}
    \begin{subfigure}{\textwidth}
        \includegraphics[width=.3\textwidth]{"src/chapters/07/img/Beautil_twenty_sequence_0".pdf}
        \includegraphics[width=.3\textwidth]{"src/chapters/07/img/Beautil_twenty_sequence_1".pdf}
        \includegraphics[width=.3\textwidth]{"src/chapters/07/img/Beautil_twenty_sequence_0_78".pdf}
    \end{subfigure}
    \begin{subfigure}{\textwidth}
        \includegraphics[width=.3\textwidth]{"src/chapters/07/img/Beautil_basic_sequence_0".pdf}
        \includegraphics[width=.3\textwidth]{"src/chapters/07/img/Beautil_basic_sequence_1".pdf}
        \includegraphics[width=.3\textwidth]{"src/chapters/07/img/Beautil_basic_sequence_0_78".pdf}
    \end{subfigure}
\end{figure}

\begin{figure}[!htbp]
    \begin{subfigure}{\textwidth}
        \includegraphics[width=.3\textwidth]{"src/chapters/07/img/Beautil_lstm_classification_0".pdf}
        \includegraphics[width=.3\textwidth]{"src/chapters/07/img/Beautil_lstm_classification_1".pdf}
        \includegraphics[width=.3\textwidth]{"src/chapters/07/img/Beautil_lstm_classification_0_78".pdf}
    \end{subfigure}
    \begin{subfigure}{\textwidth}
        \includegraphics[width=.3\textwidth]{"src/chapters/07/img/Beautil_top_twenty_classification_0".pdf}
        \includegraphics[width=.3\textwidth]{"src/chapters/07/img/Beautil_top_twenty_classification_1".pdf}
        \includegraphics[width=.3\textwidth]{"src/chapters/07/img/Beautil_top_twenty_classification_0_78".pdf}
    \end{subfigure}
    \begin{subfigure}{\textwidth}
        \includegraphics[width=.3\textwidth]{"src/chapters/07/img/Beautil_twenty_classification_0".pdf}
        \includegraphics[width=.3\textwidth]{"src/chapters/07/img/Beautil_twenty_classification_1".pdf}
        \includegraphics[width=.3\textwidth]{"src/chapters/07/img/Beautil_twenty_classification_0_78".pdf}
    \end{subfigure}
    \begin{subfigure}{\textwidth}
        \includegraphics[width=.3\textwidth]{"src/chapters/07/img/Beautil_basic_classification_0".pdf}
        \includegraphics[width=.3\textwidth]{"src/chapters/07/img/Beautil_basic_classification_1".pdf}
        \includegraphics[width=.3\textwidth]{"src/chapters/07/img/Beautil_basic_classification_0_78".pdf}
    \end{subfigure}
\end{figure}


% THE BIG TABLE V.K. SUGGESTED HERE
\section{Chapter Summary}
