\begin{proof}
The optimal behaviour of a memory-one strategy player
\(p^* \in \R_{[0, 1]} ^ 4\)
against a set of \(N\) opponents \(\{q^{(1)}, q^{(2)}, \dots, q^{(N)} \}\)
for \(q^{(i)} \in \R_{[0, 1]} ^ 4\) is established by:

\[p^* = \textnormal{argmax}\left(\sum\limits_{i=1} ^ N  u_q(p)\right), \ p \in S_q,\]

where \(S_q\) is given by (\ref{eq:s_q_set}).

The optimisation problem of (\ref{eq:mo_tournament_optimisation}) can be
written as:

\begin{equation}\label{eq:mo_tournament_optimisation_standard}
    \begin{aligned}
    \max_p: & \ \sum_{i=1} ^ {N} {u_q}^{(i)} (p)
    \\
    \text{such that}: p_i & \leq 1 \text{ for } \in \{1, 2, 3, 4\} \\
    - p_i & \leq 0 \text{ for } \in \{1, 2, 3, 4\} \\
    \end{aligned}
\end{equation}

The optimisation problem has two inequality constraints and regarding the optimality
this means that:

\begin{itemize}
    \item either the optimum is away from the boundary of the optimization domain, and so the constraints plays no role;
    \item or the optimum is on the constraint boundary.
\end{itemize}

Thus, the following three cases must be considered:

\textbf{Case 1:} The solution is on the boundary and any of the possible
combinations for $p_i \in \{0, 1\}$ for $i \in \{1, 2, 3, 4\}$ are candidate
optimal solutions.

\textbf{Case 2:} The optimum is away from the boundary of the optimization domain
and the interior solution $p^*$ necessarily satisfies the condition
\(\frac{d}{dp} \sum\limits_{i=1} ^ N  u_q(p^*) = 0\).

\textbf{Case 3:} The optimum is away from the boundary of the optimization domain
but some constraints are equalities. The candidate solutions in this case
are any combinations of $p_j \in \{0, 1\} \quad \text{and} \quad \frac{d}{dp_k} 
\sum\limits_{i=1} ^ N  u_q^{(i)}(p) = 0$ 
forall $ j \in J \text{ \& } k \in K \text{ forall } J, K
\text{ where } J \cap K = \O \text{ and } J \cup K = \{1, 2, 3, 4\}.$

Combining cases 1-3 a set of candidate solution is constructed as:

\begin{equation*}
    S_q =
    \left\{p \in \mathbb{R} ^ 4 \left|
        \begin{aligned}
            \bullet\quad p_j \in \{0, 1\} & \quad \text{and} \quad \frac{d}{dp_k} 
            \sum\limits_{i=1} ^ N  u_q^{(i)}(p) = 0
            \quad \text{for all} \quad j \in J \quad \&  \quad k \in K  \quad \text{for all} \quad J, K \\
            & \quad \text{where} \quad J \cap K = \O \quad
            \text{and} \quad J \cup K = \{1, 2, 3, 4\}.\\
            \bullet\quad  p \in \{0, 1\} ^ 4
        \end{aligned}\right.
    \right\}.
\end{equation*}

This set is denoted as $S_q$ and the optimal solution to
(\ref{eq:mo_tournament_optimisation}) is the point from $S_q$ for which the
utility is maximised.

% The Lagrangian\cite{bertsekas2014} of (\ref{eq:mo_tournament_optimisation_standard})
% is then given by,

% \begin{align}
% L(p_1, p_2, p_3, p_4, \lambda_1, \lambda_2, \dots, \lambda_8) = \sum\limits_{i=1} ^ N  u_q(p)
% + \lambda_1 (p_1 - 1) + \lambda_2 (p_2 - 1) + \lambda_3 (p_3 - 1) + \lambda_4 (p_4 - 1) + \\
% \lambda_5( - p_1) + \lambda_6 (- p_2) + \lambda_7 (- p_3) + \lambda_8 (- p_4)
% \end{align}

% This gives the following Karush-Kuhn-Tucker~\cite{Giorgi2016} conditions:

% \begin{align}
% \frac{d\sum\limits_{i=1} ^ N  u_q(p)}{dp_1} + \lambda_1 -\lambda_5 = 0 \\
% \frac{d\sum\limits_{i=1} ^ N  u_q(p)}{dp_2} + \lambda_2 -\lambda_6 = 0 \\
% \frac{d\sum\limits_{i=1} ^ N  u_q(p)}{dp_3} + \lambda_3 -\lambda_7 = 0 \\
% \frac{d\sum\limits_{i=1} ^ N  u_q(p)}{dp_4} + \lambda_4 -\lambda_8 = 0 \\
% \lambda_i (p_i - 1) = 0 \text{ for } i \in \{1, 2, 3, 4\} \\
% -\lambda_5 p_1 = 0 \\
% -\lambda_6 p_2 = 0 \\
% -\lambda_7 p_3 = 0 \\
% -\lambda_8 p_4 = 0
%     \end{align}

% There are eight complementarity conditions (37) - (41), thus a total of 16 cases
% to be checked.

% \textbf{Case 1:} \(\lambda_1 = \lambda_2 = \dots = \lambda_8 = 0\). The best response
% is given by the roots of the partial derivatives
% \(\frac{d\sum\limits_{i=1} ^ N  u_q(p)}{dp} = 0\).

% \textbf{Case 2:} \(\lambda_1 = \lambda_2 = \lambda_3 = \lambda_4 = 0\) and
% \(\lambda_5 \neq 0, \lambda_6 \neq 0,  \lambda_7 \neq 0, \lambda_8 \neq 0\). The
% best response is given by \(p_1 = p_2 = p_3 = p_4 = 0\).

% \textbf{Case 3:} \(\lambda_5 = \lambda_6 = \lambda_7 = \lambda_8 = 0\) and
% \(\lambda_1 \neq 0, \lambda_2 \neq 3,  \lambda_4 \neq 0, \lambda_5 \neq 0\). The
% best response is given by \(p_1 = p_2 = p_3 = p_4 = 1\).


\end{proof}