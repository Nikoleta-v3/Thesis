\begin{proof}

The utility \(u_q(p)\) is non concave because the concavity condition fails for at
least one pair of points see Appendix~\ref{appendix:non_concave}.

Furthermore, regarding the numerator and denominator of \(u_q(p)\)
in~\cite{Anton2014} it is stated that a quadratic form will be concave if and
only if it's symmetric matrix is semi-negative definite. A matrix \(A\) is
semi-negative definite if:

\begin{equation}\label{def:semi_negative}
|A|_i \leq 0 \text{ for } i \text{ is odd and } |A|_i \geq 0  \text{ for } i
\text{ is even,}
\end{equation}

where \(|A|_i\) is the eigenvalues of the submatrix \(A_i\).

For (\ref{eq:optimisation_quadratic}), neither \(\frac{1}{2}pQp^T + cp + a\)
or \(\frac{1}{2}p\bar{Q}p^T + \bar{c}p + \bar{a}\) are concave because for an even \(i=2\):

\[|Q|_2 = - \left(q_{1} - q_{3}\right)^{2} \left(q_{2} - 5 q_{4} - 1\right)^{2} \text{and}\]
\[|\bar{Q}|_2 =- \left(q_{1} - q_{3}\right)^{2} \left(q_{2} - q_{4} - 1\right)^{2}\]

are negative.

Moreover, for a quadratic to be strictly positive it has to be positive definite.
A quadratic form is positive definite iff every eigenvalue of is positive,
however, \(\frac{1}{2}p\bar{Q}p^T + \bar{c}p + \bar{a}\) is not positive definite
because:

\[|\bar{Q}|_2 =- \left(q_{1} - q_{3}\right)^{2} \left(q_{2} - q_{4} - 1\right)^{2}\]

is negative.
\end{proof}