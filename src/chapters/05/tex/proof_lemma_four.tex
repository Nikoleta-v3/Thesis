\begin{proof}
The optimal behaviour of a memory-one strategy player
\(p^* \in \R_{[0, 1]} ^ 4\)
against a set of \(N\) opponents \(\{q^{(1)}, q^{(2)}, \dots, q^{(N)} \}\)
for \(q^{(i)} \in \R_{[0, 1]} ^ 4\) is established by:

\[p^* = \textnormal{argmax}\left(\sum\limits_{i=1} ^ N  u_q(p)\right), \ p \in S_q,\]

where \(S_q\) is given by (\ref{eq:s_q_set}).

The optimisation problem of (\ref{eq:mo_tournament_optimisation}) can be
written as:

\begin{equation}\label{eq:mo_tournament_optimisation_standard}
    \begin{aligned}
    \max_p: & \ \sum_{i=1} ^ {N} {u_q}^{(i)} (p)
    \\
    \text{such that}: p_i & \leq 1 \text{ for } \in \{1, 2, 3, 4\} \\
    - p_i & \leq 0 \text{ for } \in \{1, 2, 3, 4\} \\
    \end{aligned}
\end{equation}

The optimisation problem has two inequality constraints and regarding the optimality
this means that:

\begin{itemize}
    \item either the optimum is away from the boundary of the optimization domain, and so the constraints plays no role;
    \item or the optimum is on the constraint boundary.
\end{itemize}

Thus, the following three cases must be considered:

\textbf{Case 1:} The solution is on the boundary and any of the possible
combinations for $p_i \in \{0, 1\}$ for $i \in \{1, 2, 3, 4\}$ are candidate
optimal solutions.

\textbf{Case 2:} The optimum is away from the boundary of the optimization domain
and the interior solution $p^*$ necessarily satisfies the condition
\(\frac{d}{dp} \sum\limits_{i=1} ^ N  u_q(p^*) = 0\).

\textbf{Case 3:} The optimum is away from the boundary of the optimization domain
but some constraints are equalities. The candidate solutions in this case
are any combinations of $p_j \in \{0, 1\} \quad \text{and} \quad \frac{d}{dp_k} 
\sum\limits_{i=1} ^ N  u_q^{(i)}(p) = 0$ 
forall $ j \in J \text{ \& } k \in K \text{ forall } J, K
\text{ where } J \cap K = \O \text{ and } J \cup K = \{1, 2, 3, 4\}.$

Combining cases 1-3 a set of candidate solution is constructed as:

\begin{equation*}
    \resizebox{\linewidth}{!}{ $
    S_q =
    \left\{p \in \mathbb{R} ^ 4 \left|
        \begin{aligned}
            \bullet\quad p_j \in \{0, 1\} & \quad \text{and} \quad \frac{d}{dp_k} 
            \sum\limits_{i=1} ^ N  u_q^{(i)}(p) = 0
            \quad \text{for all} \quad j \in J \quad \&  \quad k \in K  \quad \text{for all} \quad J, K \\
            & \quad \text{where} \quad J \cap K = \O \quad
            \text{and} \quad J \cup K = \{1, 2, 3, 4\}.\\
            \bullet\quad  p \in \{0, 1\} ^ 4
        \end{aligned}\right.
    \right\}.
    $}
\end{equation*}

The derivative of \(\sum\limits_{i=1} ^ N  u_q^{(i)}(p)\) calculated using
the following property (see~\cite{Abadir2005} for details):

\begin{equation}\label{eq:first_derivative_property}
\frac{d x A x^T}{dx} =  2Ax.
\end{equation}

Using property~(\ref{eq:first_derivative_property}):

\begin{equation}\label{eq:quadratics_derivatives}
\frac{d}{dp} \frac{1}{2}pQp^T + cp + a = pQ + c \text{ and } \frac{d}{dp} \frac{1}{2}p\bar{Q}p^T + \bar{c}p + \bar{a} = p\bar{Q} + \bar{c}.
\end{equation}

Note that the derivative of \(cp\) is \(c\) and the constant disappears.
Combining these it can be proven that:

\begin{equation*}
\resizebox{\textwidth}{!}{$ \begin{split}
\frac{d}{dp} \sum\limits_{i=1} ^ N  u_q^{(i)}(p) & = \sum\limits_{i=1} ^ N \frac{\frac{d}{dp}(\frac{1}{2}pQ^{(i)}p^T + c^{(i)}p + a^{(i)} )(\frac{1}{2}p\bar{Q^{(i)}}p^T + \bar{c^{(i)}}p + \bar{a^{(i)}}) -
\frac{d}{dp}(\frac{1}{2}p\bar{Q^{(i)}}p^T + \bar{c^{(i)}}p + \bar{a^{(i)}})(\frac{1}{2}pQ^{(i)}p^T + c^{(i)}p + a^{(i)})}{(\frac{1}{2}p\bar{Q^{(i)}}p^T + \bar{c^{(i)}}p + \bar{a^{(i)}})^2} \\
& = \sum\limits_{i=1} ^ N \frac{(pQ^{(i)} + c^{(i)} +)(\frac{1}{2}p\bar{Q^{(i)}}p^T + \bar{c^{(i)}}p + \bar{a^{(i)}})}{(\frac{1}{2}p\bar{Q^{(i)}}p^T + \bar{c^{(i)}}p + \bar{a^{(i)}})^2} -
 \frac{(p\bar{Q^{(i)}}+ \bar{c^{(i)}})(\frac{1}{2}pQ^{(i)}p^T + c^{(i)}p + a^{(i)})}{(\frac{1}{2}p\bar{Q^{(i)}}p^T + \bar{c^{(i)}}p + \bar{a^{(i)}})^2}
\end{split}$}
\end{equation*}

For \(\frac{d}{dp} \sum\limits_{i=1} ^ N  u_q(p)\) to equal zero then:

{\scriptsize
\begin{align}\label{eq:polynomials_roots}
    \displaystyle\sum\limits_{i=1} ^ {N}
    \left(pQ^{(i)} + c^{(i)}\right) \left(\frac{1}{2} p\bar{Q}^{(i)} p^T + \bar{c}^{(i)} p + \bar{a}^ {(i)}\right)
    - \left(p\bar{Q}^{(i)} + \bar{c}^{(i)}\right) \left(\frac{1}{2} pQ^{(i)} p^T + c^{(i)} p + a^ {(i)}\right)
    & = 0, \quad {while} \\
    \displaystyle\sum\limits_{i=1} ^ {N} \frac{1}{2} p\bar{Q}^{(i)} p^T + \bar{c}^{(i)} p + \bar{a}^ {(i)} & \neq 0.
\end{align}}

The optimal solution to Equation~\ref{eq:mo_tournament_optimisation} is the point
from $S_q$ for which the utility is maximised.

\end{proof}