\chapter{Best Response Sequences in the Iterated Prisoner's Dilemma}\label{chapter:best_response_sequence}

\begin{center}
    The research reported in this Chapter has been carried with:

    Associated data set: \cite{Glynatsi2020_sequences} \\
    Axerod-Python library version: 4.2.0 \\ \vspace{.5cm}
\end{center}

\section{Introduction}

In this Chapter best response strategies are explored in the form
of static sequences of moves in order to generate a large data set of best
response sequences to a collection of opponents.

These best responses are not estimated explicitly but heuristically using a genetic
algorithm. The data set of best response sequences is generated by considering:

\begin{itemize}
    \item IPD matches, and subsequently sequences, of 205 turns.
    \item Opponents from a collection of \numberofstrategiesbestsequences strategies
    available in the APL.
\end{itemize}

The purpose of the data set is to be used as training data in
Chapter~\ref{chapter:lstm} where an IPD player is trained by a recurrent neural
network. The Chapter is structured as follow:

\begin{itemize}
    \item section~\ref{section:ipd_as_sequences} formalises the formulation of
    the Chapter and the use of sequences to express an player in an IPD match
    against a give strategy.
    \item section~\ref{section:genetic_algorithm} details the reinforcement
    technique, specifically the genetic algorithm, applied to identify best response
    sequences.
    \item section~\ref{section:generating_sequences} details the usage of the
    genetic algorithm and a specific player from APL called Cycler to generate a
    broad number of best response sequences to a collection of
    \numberofstrategiesbestsequences strategies.
\end{itemize}

\section{Iterated Prisoner Dilemma Strategies as Sequences}\label{section:ipd_as_sequences}

In a finite \(N\) round IPD match, a participants can be expressed by a sequence:

\begin{equation}
    S \in \{C, D\} ^ n
\end{equation}

where \(n\) is the length of the sequence and \(1 \leq n \leq N\).

Strategies that use sequences to play an IPD are already established in the
literature~\cite{Beaufils1997, Knight2016, Li2011, Mittal2009}. More
specifically, the aforementioned works have considered strategies that
periodically play a sequence \(S\). These strategies have been refereed to as
Periodic Players \(S\)~\cite{Li2011, Mittal2009} and as Cycler
\(S\)~\cite{Knight2016} where \(S \in \{CD, DC, CCD, DDC\}\), and thus \(n \in
\{2, 3\}\). Similarly, this Chapter will consider such strategies but
strictly for \(n = N\).

Consider a match of \(N = 10\) between the strategy\(S = \{D, D, D, C, C, C, D,
D, C, C\}\) and a Cooperator. The match between the two strategies is captured
by:

\begin{table}[htb]
\centering
\begin{tabular}{cccccccccccc}
    & \textbf{1} & \textbf{2} & \textbf{3} & \textbf{4}  & \textbf{5} & \textbf{6} & \textbf{7} & \textbf{8}  & \textbf{9} & \textbf{10} & \(U\) \\ \midrule
    \(S\) & \(D\) & \(D\) & \(D\) & \(C\) & \(C\) & \(C\) & \(D\) & \(D\) & \(C\) & \(C\) & 4.0 \\
    Cooperator & \(C\) & \(C\) & \(C\) & \(C\) & \(C\) & \(C\) & \(C\) & \(C\) & \(C\) & \(C\) & 1.5 \\ \bottomrule
\end{tabular}
\caption{Match between \(S = \{D, D, D, C, C, C, D, D, C, C\}\) and Cooperator.}\label{table:s_vs_cooperator}
\end{table}

where, \(U\) is the average score per turn the strategies scored. A sequence
strategy \(S\) can play against strategies that react to the history, as demonstrated
by:

\begin{table}[htb]
\centering
\begin{tabular}{cccccccccccc}
    & \textbf{1} & \textbf{2} & \textbf{3} & \textbf{4}  & \textbf{5} & \textbf{6} & \textbf{7} & \textbf{8}  & \textbf{9} & \textbf{10} & \(U\) \\ \midrule
    \(S\) & \(D\) & \(D\) & \(D\) & \(C\) & \(C\) & \(C\) & \(D\) & \(D\) & \(C\) & \(C\) & 2.2 \\
    Tit For Tat & \(C\) & \(D\) & \(D\) & \(D\) & \(C\) & \(C\) & \(C\) & \(D\) & \(D\) & \(C\) & 2.2 \\ \bottomrule
\end{tabular}
\caption{Match between \(S = \{D, D, D, C, C, C, D, D, C, C\}\) and Tit For Tat.}\label{table:s_vs_tft}
\end{table}

and against sophisticated strategies, such as Random with a probability \(0.5\)
of cooperating at each turn:

\begin{table}[htb]
\centering
\begin{tabular}{cccccccccccc}
    & \textbf{1} & \textbf{2} & \textbf{3} & \textbf{4}  & \textbf{5} & \textbf{6} & \textbf{7} & \textbf{8}  & \textbf{9} & \textbf{10} & \(U\) \\ \midrule
    \(S\) & \(D\) & \(D\) & \(D\) & \(C\) & \(C\) & \(C\) & \(D\) & \(D\) & \(C\) & \(C\) & 2.4 \\
    Random & \(C\) & \(D\) & \(D\) & \(C\) & \(C\) & \(C\) & \(D\) & \(D\) & \(C\) & \(C\) & 1.9 \\ \bottomrule
\end{tabular}
\caption{Match between \(S = \{D, D, D, C, C, C, D, D, C, C\}\) and Random.}\label{table:s_vs_random}
\end{table}

Note that the game of Table~\ref{table:s_vs_random} only captures a single
behaviour of Random against \(S = \{D, D, D, C, C, C, D, D, C, C\}\). Random is
a stochastic strategy, and thus the strategy's series of actions can differ even
when the history of plays is the same. For example a match between \(S = \{D, D,
D, C, C, C, D, D, C, C\}\) and Random can also be capture by:

\begin{table}[htb]
    \centering
    \begin{tabular}{cccccccccccc}
        & \textbf{1} & \textbf{2} & \textbf{3} & \textbf{4}  & \textbf{5} & \textbf{6} & \textbf{7} & \textbf{8}  & \textbf{9} & \textbf{10} & \(U\) \\ \midrule
        \(S\) & \(D\) & \(D\) & \(D\) & \(C\) & \(C\) & \(C\) & \(D\) & \(D\) & \(C\) & \(C\) & 1.6 \\
        Random & \(D\) &\(D\) &\(C\) &\(C\) &\(D\) &\(D\) &\(D\) &\(C\) &\(D\) &\(D\) & 2.6 \\ \bottomrule
    \end{tabular}
\end{table}

In section~\ref{section:generating_sequences} computer seeding will be defined.
The usage of computer seeding will allow this to capture and reproduce
several instances/behaviour of stochastic strategies.

A best response sequence to a given opponent \(q\) is the sequence for which th
average score per turn is maximised. More specifically,

\begin{equation}\label{eq:best_response}
    \begin{aligned}
    \max_{S^*}: & U_{(S^*, q)}
    \end{aligned}
\end{equation}

where \(q\) is not a memory-one opponent, but can be any IPD strategy.

Identifying best responses to some opponents can be a trivial problem, for
example the best response sequence for \(N\) turns against Cooperator is
\(\{D\}^N\), however, for most strategies identifying best responses is a
complex problem.

For this reason, best response sequences are not manually identified but instead
a heuristic method called genetic algorithm is used. A background on genetic
algorithms and the exact algorithm that is used in this Chapter are detailed in
section~\ref{section:genetic_algorithm}.

\section{Genetic Algorithm}\label{section:genetic_algorithm}

A genetic algorithm (GA) is a heuristic inspired by the process of natural
selection that belongs to the larger class of evolutionary algorithms. As stated
in~\cite{Whitley1994} GAs encode a potential solution to a
specific problem on a simple chromosome-like data structure, and apply
recombination operators to these structures in such a way as to preserve
critical information. GAs are often viewed as function
optimizers, although the range of problems to which they have been
applied~\cite{Hou1994, Jones1997, Yang1998} is quite broad.

An implementation of a GA begins with a population \(P\) of
potential solutions, a number of generations \(G \in \N\) and a cut-off or
bottleneck \(b < |P|\). At each generation the algorithm scores and potentially
removes each member of the population \(p_i \in P\). This is done by using a
mapping from a member of the population to an ordered set based on a evaluation
function \(f\), such that \(f(p_i) \to \R\). At the conclusion of each
generation the top \(b\) ranking members (or proportion of members) by score are
be kept and the rest are discarded. By doing so, critical information regarding
the successful candidates is preserved and the population rebuilds on it using a
series of \textit{crossovers} and \textit{mutations}.

\begin{itemize}
    \item Crossovers take in 2 members of the population and return a new member
    based on some ``genes'' of the 2 selected members (commonly refereed to as parents).
    \item Mutations allow a change in a single member of the population. Mutation
    is commonly associated with a probability \(p_m\) which is the probability
    of a mutation happening, either at the individual or at each gene of the
    individual.
\end{itemize}

GA has the ability to avoid being trapped in local optimal solution like
traditional methods. This ability is due to the properties of crossover and
mutation. Crossover adds variance in the population by taking properties of both
parent members and introducing the evaluation of their combination to the
solution space. The algorithm can search multiple local optimal at once with its
population and start to identify the global optimal as members become more
optimized. Moreover, if an a member if trapped in a local optimal with its
fitness score, a mutation of a certain feature could potentially move their
position in the solution space to better local optimal.

A diagrammatical representation of a generic GA is given in Figure~\ref{fig:ga_flow_diagram}.

\begin{figure}[!htbp]
    \centering
    \includestandalone[width=\textwidth]{src/chapters/06/ga_flow_diagram}
    \caption{Generic flow diagram of a GA.}\label{fig:ga_flow_diagram}
\end{figure}

The purpose of a GA in this Chapter is to estimate the best response sequence to
a given opponent. Consequently, the members of the population correspond to
sequences \(S\) of length \(N\), and the evaluation function corresponds to the
average score per turn of the sequence against it's opponent. The detailed GA of
this work is given by Algorithm~\ref{algorithm:genetic_algorithm}.

\begin{algorithm}[!htbp]
    \SetAlgoLined
    \KwIn{\(q, N, b, p_m, G, s\)}
    \KwOut{Population at the last generation and the members' scores}
     \Begin{
     create initial population (Algorithm~\ref{algorithm:initial_population}) of members \(S\), where \(|S| = N\) and \(|P| = s\)\\
     \While{\(g_{i} < G\)}{
        score each member based on \(U_{S, q}\) for \(S \in P\) \\
        sort population based on scores \\
        keep \(b\) top members \\
        \While{\(|P|\) \(< s\)}{
            select 2 random members \\
            use members to create new member through crossover\\
            \For{gene in new member}{
            mutate gene with probability \(p_m\)}
            add new member to population\\
            }
     }}
     \caption{Get population of potential best response sequences}\label{algorithm:genetic_algorithm}
\end{algorithm}

The initial population is generated using Algorithm~\ref{algorithm:initial_population}.
Using a starting population of random guesses is a generally common approach in
the GA literature~\cite{Hou1994}. However, there is efficiency in using non
random starting populations~\cite{Drezner2005}. In~\cite{Osaba2014} it is
stated that the excessive use of heuristic initialization functions
could decrease the exploration capacity of a GA, trapping the population in
local optimums quickly, and therefore, the key is to maintain a balance between
individuals initialised by functions, and individuals generated randomly.
Here member that have been initialised by functions are not considered. However,
in the IPD the best responses to a board number of well established strategies
are either of the deterministic plays of always cooperating or defecting.
Thus, both of these are included in the initial population and the rest of
the population is filled with a combination of these two strategies.

\begin{algorithm}[!htbp]
        \SetAlgoLined
        \KwIn{\(s, N\)}
        \KwOut{A population of size \(s\).}
    
        \Begin{
        set of cuts $\gets$ \(s\) evenly spaced numbers over \([1, N]\) \\
        \For{\(c \in\) set of cuts}{
            first new member $\gets$  \(\{C\}^{c} \cup \{D\}^{N-c}\) \\
            second new member $\gets$  \(\{D\}^{c} \cup \{C\}^{N-c}\) \\
            add both members to population
        }
            }
     \caption{Create initial population of individuals \(S\)}\label{algorithm:initial_population}
\end{algorithm}

The crossover between two members occurs by randomly selecting a crossover point
\(< N\), and the new member inheriting the genes right to the crossover point from
the first parent and the left genes to the crossover point from the second
parent. A mutation is applied to new member before they are added to the
population. During mutation there is probability \(p_m\) that each gene of the
new member is flipped.

The GA of this Chapter has been implemented in the programming language Python
and it has been organised into a open source package called
\mintinline{python}{sequence_sensei} available at. %TODO archive
Figure~\ref{fig:get_initial_population} demonstrates the implementation of
Algorithm~\ref{algorithm:initial_population} in the package. Moreover,
Figure~\ref{fig:get_initial_population_example} gives a usage example of
generating a starting population of a given size using
\mintinline{python}{sequence_sensei}.
The bottom members of the populations are the sequences \(\{1\}^N\) and
\(\{0\}^N\), where \(0 \to D\) and \(1 \to C\), as expected. In
Figure~\ref{fig:get_initial_population_example} it is also illustrated how APL
can map binary numbers to the IPD actions.

\begin{figure}[!htbp]
\begin{sourcepy}
import numpy as np
def get_initial_population(half_size_of_population, sequence_length):
    """
    Generates an initial population of sequences. Note that the length
    of the population which is being generated is 2 * half_size_of_population.
    """
    cuts = np.linspace(1, sequence_length, half_size_of_population, dtype=int)
    sequences = []
    for cut in cuts:
        sequences.append(
            [1 for _ in range(cut)] + [0 for _ in range(sequence_length - cut)]
        )
        sequences.append(
            [0 for _ in range(cut)] + [1 for _ in range(sequence_length - cut)]
        )

    return sequences
\end{sourcepy}
\caption{Source code of the \mintinline{python}{get_initial_population} function
implemented in \mintinline{python}{sequence_sensei}.}\label{fig:get_initial_population}
\end{figure}

\begin{figure}[!htbp]
    \begin{usagepy}
>>> import sequence_sensei as ss
>>> initial_population = ss.get_initial_population(half_size_of_population=5, sequence_length=8)
>>> initial_population
[[1, 0, 0, 0, 0, 0, 0, 0],
 [0, 1, 1, 1, 1, 1, 1, 1],
 [1, 1, 0, 0, 0, 0, 0, 0],
 [0, 0, 1, 1, 1, 1, 1, 1],
 [1, 1, 1, 1, 0, 0, 0, 0],
 [0, 0, 0, 0, 1, 1, 1, 1],
 [1, 1, 1, 1, 1, 1, 0, 0],
 [0, 0, 0, 0, 0, 0, 1, 1],
 [1, 1, 1, 1, 1, 1, 1, 1],
 [0, 0, 0, 0, 0, 0, 0, 0]]

 >>> import axelrod as axl
 >>> [[axl.Action(gene) for gene in member] for member in initial_population]
[[C, D, D, D, D, D, D, D],
 [D, C, C, C, C, C, C, C],
 [C, C, D, D, D, D, D, D],
 [D, D, C, C, C, C, C, C],
 [C, C, C, C, D, D, D, D],
 [D, D, D, D, C, C, C, C],
 [C, C, C, C, C, C, D, D],
 [D, D, D, D, D, D, C, C],
 [C, C, C, C, C, C, C, C],
 [D, D, D, D, D, D, D, D]]
    \end{usagepy}
    \caption{Example of using \mintinline{python}{get_initial_population} to
    generate a population of \(s=10\) and \(N=8\).}\label{fig:get_initial_population_example}
\end{figure}

The implementations of the crossover and mutation properties are given in Figure~\ref{fig:crossover_mutation},
futhermore, an example of a crossover between a Cooperator and an Alternator and the
mutation of the crossover's result is given in Figure~\ref{fig:crossover_mutation}.

\begin{figure}[!htbp]
\begin{sourcepy}
import random

def crossover(sequence_one, sequence_two):
    sequence_length = len(sequence_one)
    crossover_point = random.randint(0, sequence_length)

    return sequence_one[:crossover_point] + sequence_two[crossover_point:]

def mutation(gene, mutation_probability):
    if random.random() < mutation_probability:
        return abs(gene - 1)
    return gene
\end{sourcepy}
\caption{Source code of crossover and mutation properties in \mintinline{python}{sequence_sensei}.}\label{fig:crossover_mutation}
\end{figure}

\begin{figure}[!htbp]
\begin{usagepy}
>>> import random
>>> import sequence_sensei as ss

>>> turns = 10
>>> cooperator = [1 for _ in range(turns)]
>>> alternator = [i % 2 for i in range(turns)]

>>> random.seed(1)
>>> new_member = ss.crossover(cooperator, alternator)
>>> new_member
[1, 1, 0, 1, 0, 1, 0, 1, 0, 1]

>>> random.seed(1)
>>> [ss.mutation(gene, mutation_probability=0.05) for gene in new_member]
[1, 1, 0, 1, 0, 1, 0, 1, 0, 0]

\end{usagepy}
\caption{Example of crossover between a Cooperator and an Alternator, and an example
of the mutation property.}\label{fig:crossover_mutation}
\end{figure}

The main function implemented in \mintinline{python}{sequence_sensei} for
performing a GA is \mintinline{python}{evolved} function. The function has
several input arguments which correspond to the input as given in
Algorithm~\ref{algorithm:genetic_algorithm}. Details for the parameters' values
and the scoring process of the populations, and the process of generating a best
response sequence data set are presented in
section~\ref{section:generating_sequences}.

\section{Generating Best Response Sequences}\label{section:generating_sequences}

In order to generate the data set of best response sequences a collection of
\numberofstrategiesbestsequences strategies is consider for opponents. The
results of this Chapter relay on the APL project to access this collection of
strategies, and moreover, to simulate the IPD games and to calculate the score
of the sequence strategies.

The APL projects contains a strategy called Cycler which plays a given sequence
\(S\). The Cycler strategy is considered to simulate the matches between the
\numberofstrategiesbestsequences strategies and the potential best response
sequences. An example of using the strategy Cycler is given in
Figure~\ref{fig:apl_simulations_cycler}. Figure~\ref{fig:apl_simulations_cycler}
illustrates how the matches of Tables~\ref{table:s_vs_cooperator}
and~\ref{table:s_vs_tft} can be simulated using APL, and moreover how the score
of each strategy and is accessible once the match has been performed.

\begin{figure}[!htbp]
    \begin{usagepy}
>>> import axelrod as axl

>>> players = [axl.Cycler('DDDCCCDDCC'), axl.Cooperator()]
>>> match = axl.Match(players, turns=10)
>>> match.play()
[(D, C),
 (D, C),
 (D, C),
 (C, C),
 (C, C),
 (C, C),
 (D, C),
 (D, C),
 (C, C),
 (C, C)]

>>> match.final_score_per_turn()
(4.0, 1.5)

>>> players = [axl.Cycler('DDDCCCDDCC'), axl.TitForTat()]
>>> match = axl.Match(players, turns=10)
>>> match.play()
[(D, C),
 (D, D),
 (D, D),
 (C, D),
 (C, C),
 (C, C),
 (D, C),
 (D, D),
 (C, D),
 (C, C)]

 >>> match.final_score_per_turn()
 (2.2, 2.2)
    \end{usagepy}
\caption{Simulating matches using a Cycler to play a given sequence.}\label{fig:apl_simulations_cycler}
\end{figure}

The \numberofstrategiesbestsequences strategies used in this Chapter are found
in the Appendix alongside a citation of their paper of origin.

From these \numberofstrategiesbestsequences strategies, 62 are stochastic and
135 are deterministic. As it has been established, the outcome of a match
between two deterministic strategies never changes, as longs as the environment
does not include noise. In comparison, in matches which include even a single a
stochastic strategy the outcomes can be significantly different. In order to
capture distinct behaviours of stochastic strategies, and to be able to
reproduce the results of a match with a stochastic strategy this work considers
\textit{computer seeding}. In Python a computer seed is used to initialize the
pseudorandom number generator. Seeds are set before generating a random number,
and if the same seed is used on initialisation then the random output remains
the same. Thus, as long as a match is being seeded the behaviour of a stochastic
strategy can be reproduced, and different seed lead to different series of
random numbers.

Consider the example of Random of Table~\ref{table:s_vs_random}. In
Figure~\ref{fig:random_apl_example} it is demonstrated how by seeding before the
random calls the actions of the strategy are different.

\begin{figure}[!htbp]
    \begin{usagepy}
>>> players = [axl.Cycler('DDDCCCDDCC'), axl.Random()]
>>> for seed in range(5):
...   axl.seed(seed)
...   match = axl.Match(players, turns=10)
...   actions = match.play()
...   print(actions, match.final_score_per_turn())
...   print("================================================================================")
[(D, D), (D, D), (D, C), (C, C), (C, D), (C, C), (D, D), (D, C), (C, C), (C, D)] (2.2, 2.2)
================================================================================
[(D, C), (D, D), (D, D), (C, C), (C, C), (C, C), (D, D), (D, D), (C, C), (C, C)] (2.4, 1.9)
================================================================================
[(D, D), (D, D), (D, C), (C, C), (C, D), (C, D), (D, D), (D, C), (C, D), (C, D)] (1.6, 2.6)
================================================================================
[(D, C), (D, D), (D, C), (C, D), (C, D), (C, C), (D, C), (D, D), (C, C), (C, C)] (2.6, 2.1)
================================================================================
[(D, C), (D, C), (D, C), (C, C), (C, C), (C, C), (D, D), (D, D), (C, D), (C, C)] (2.9, 1.9)
================================================================================
    \end{usagepy}
\caption{Example code of using seeding to generated different instances of Random.
The above code snipped will always have the same output even if the matches are
``random'' because of the seed.}\label{fig:random_apl_example}
\end{figure}

For the data generating process of this Chapter a GA is performed for each of
the \numberofstrategiesbestsequences strategies, moreover, for the 62 stochastic
strategies 10 different seeded version of them are considered, and thus the
final number of opponents is 751. The GA outputs the population at the final
population, as well as the scores associated with each members. The member that
are considered a sequences of strictly \(N=205\). A diagrammatical
representation of the data collection is given by
Figure~\ref{fig:data_generating_process_diagram}.

\begin{figure}[!htbp]
    \centering
    \includestandalone[width=.6\textwidth]{src/chapters/06/data_generating_diagram}
    \caption{Diagrammatical representation of the generating of best response
    sequences.}\label{fig:data_generating_process_diagram}
\end{figure}

A GA is performed for each of the opponents with the following parameters values,
Table~\ref{table:parameters_summary}.

\begin{table}[!htbp]
    \begin{center}
    \resizebox{.5\textwidth}{!}{
    \begin{tabular}{lllc} \toprule
    symbol            & parameter                     & parameter values \\ \midrule
    \(N\)             & number of turns               & \(205\) \\
    \(p\)             & probability of gene mutating  & \(\{0.01, 0.05, 0.1\}\)\\
    \(b\)             & bottle neck                   & \(10, 20\) \\
    \(t_{\text{max}}\)& maximum number of generations & \(2000\) \\
    \(s\)             & size of a population          & \(20, 30, 40\) \\ \bottomrule
    \end{tabular}}
    \end{center}
    \caption{The parameters of the genetic algorithm.}\label{table:parameters_summary}
\end{table}

The population at the final generation contains a series of potential best
responses. There are cases for which more than a single strategy is a best
response. This is intuitive following the work of
Chapter~\ref{chapter:memory_one}. In an IPD the utility function is not concave,
there are multiple local optima, and thus several best responses for given
matches. The total amount of best responses is 5503. The data has been archived
and it is available here.

\section{Chapter Summary}

This Chapter presented the problem of best responses in the form of sequences.
Though, seuqnce players have been studied in the literature there has not been
a big deal about them. The aim of formulating was to genereta a large data set
which will be used in the following chapter.

In sectiom~ the formulations was introdced, and in section the GA with the specific
de,aplde. The APLS 