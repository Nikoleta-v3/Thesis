\chapter{Introduction}\label{chapter:introduction}

This thesis aims to reinforce the understanding of best responses in the
Prisoner's Dilemma. The Iterated Prisoner’s Dilemma has been used for decades as
a model of behavioural interactions. In the 1984's the book ``The Evolution of
Cooperation'' ...

This thesis summarises, evaluates and builds upon previous literature to push
the bounds of our understanding of how mathematical modelling and computer
science techniques can used in the search of dominant behaviour. To summarise,
this thesis contributes to the discussion of dominant behaviour and provides new
mathematical frameworks for the continued understanding of the questions raised
throughout.

This introductory Chapter is set as follows:

\begin{itemize}
    \item section~\ref{section:introduction_prisoners_dilemma} introduces
    the Prisoner's Dilemma.
    \item section~\ref{section:introduction_brief_literature} covers a brief
    literature review.
    \item section~\ref{section:introduction_research_questions} formalises the research
    questions and sets out the structure of the thesis.
    \item section~\ref{section:introduction_software_development} presents the
    software development techniques used throughout of the thesis.
\end{itemize}

\section{Prisoner's Dilemma}\label{section:introduction_prisoners_dilemma}

Game theory was formalised in 1944~\cite{VonNeumann1944} and is the study of
interactions as \textit{games}. A game is a model of interacting
decision makers refereed to as \textit{players}. Each player has a set of
possible \textit{actions}, and the game captures the interactions of the player's
actions by allowing each player's \textit{payoffs} to be affected by the actions
of all players. More precisely, as given in~\cite{Osborne2004}, the formal
definition of a game is as follows,

\begin{definition}
A \textbf{game} consists of
\begin{itemize}
    \item a set of player
    \item for each player, a set of actions
    \item for each player, payoff functions over the set of all actions.
\end{itemize}
\end{definition}

On the most well known games is the Prisoner's Dilemma (PD) originally
described in~\cite{Flood1958}. The PD is a two player non-cooperative game which
illustrates aspects of political philosophy and morality; how selfishness will
lead to an `inefficiency' of the outcome, when selflessness can be
evolutionarily advantageous.

More specifically, in the PD each player has two actions, to either be selfless
and cooperate, denoted as (C), or to be selfish and defect, denoted as (D). Each
decision is made simultaneously and independently. Players' payoffs are
generally represented by (\ref{eq:pd_definition}). Both players receive a reward
for mutual cooperation, \(R\), and a payoff \(P\) for mutual defection. A player
that defects while the other cooperates receives a payoff of \(T\), whereas the
cooperator receives \(S\).

\begin{equation}\label{eq:pd_definition}
    S_p =
    \begin{pmatrix}
        R & S  \\
        T & P
    \end{pmatrix}
    \quad
    S_q =
    \begin{pmatrix}
        R & T  \\
        S & P
    \end{pmatrix}
\end{equation}

It is assumed that two cooperating players do better than two defecting ones,
and thus, the payoff of two cooperating players \(R\) is
larger than the payoff of two defecting players \(P\). A player, however, has the
temptation to deviate, as that player will receive a higher payoff \(T\) than
that of mutual cooperation \(R\) whilst the cooperator's payoff \(S\) is smaller than
\(P\). In consequence, the payoffs are constrained by
Equation~(\ref{eq:constrain_one}).

\begin{equation}\label{eq:constrain_one}
    T > R > P > S
\end{equation}

A second constraint is that which ensures that a social dilemma arises: the sum
of the utilities to both players is best when they both cooperate,
Equation~(\ref{eq:constrain_two}).

\begin{equation}\label{eq:constrain_two}
    2R > T + S
\end{equation}

A special case of the Prisoner's Dilemma is that of the \textit{donation game}.
In the donation game each player can cooperate by providing a benefit \(b\) to
the other player at their cost \(c\) with \(0 < c < d\). Then \(T=b, R=b-c,S=c,
\text{and } P=0\). The representation of the donation game payoff matrix is
given by~(\ref{eq:the_pd_payoffs_with_cost}).

\begin{equation}\label{eq:the_pd_payoffs_with_cost}
    S_p =
    \begin{pmatrix}
        b - c & c\\
        b & 0
    \end{pmatrix}
    \quad
    S_q =
    \begin{pmatrix}
        b - c & b  \\
        c & 0
    \end{pmatrix}
\end{equation}

This thesis uses the basic PD as given by Equation~(\ref{eq:pd_definition}).
The numerical experiments presented throughout the thesis are carried out
with the following payoff values of \(R = 3, P = 1, T = 5\) and \(S = 0\).
These are the values most commonly used in the literature~\cite{Axelrod1984, Knight2017}.

In non cooperative games the players interact in order to achieve the most
preferred outcome. A \textit{best response strategy} is a strategy
that maximises the utility of a player given a known strategy of the other
player. A solution concept commonly used in game theory is the Nash equilibrium~\cite{Nash1951}
which is a pair of best response strategies at which neither of the players has
a reason to deviate.

In the PD due to constrain (\ref{eq:constrain_one}) it never benefits a player
to cooperate. A player that cooperates receives either a payoff of \(R\) or \(S\)
depending the action of the other player, whereas if a player defects they
receive either \(T\) or \(P\), and \(T > R\) and \(P > S\). Once both
players defect neither have a reason to change their decision. Thus, in the
PD mutual defection is a Nash equilibrium and defection is the
best response strategy.

The game can be studied in a manner where prior outcomes matter. The repeated
form of the game is called the Iterated Prisoner's Dilemma (IPD) and it differs
from the original concept of a PD because participants can learn about the
behavioural tendencies of their counter party. In the IPD defecting is no
longer necessarily the dominant action, and identifying a best response is not
always trivial.

\section{Literature Review}\label{section:introduction_brief_literature}

In the 1980's Robert Axelrod studied the best way of behaving in the IPD by
running a series of computer tournaments with two collections of
strategies~\cite{Axelrod1984}. These strategies were written/submitted by
researchers. Axelrod performed an evolutionary tournament~\cite{Axelrod1981} and
two round robin tournaments~\cite{Axelrod1980a, Axelrod1980b}. The strategy that
took over the population and won both tournaments was the strategy Tit For Tat.
Axerlod's results demonstrated the robustness of the strategy in those
environments and subsequently the robustness of reciprocal behaviour. These
results however did not reassured the success of the strategy in other
environments. This became more evident as further competitions and mathematical
formulations introduced new dominating strategies in different environments. A
brief summary of selected works and their dominating strategies are given by
Table~\ref{table:tournament_refs}.

\begin{table}[htbp]
    \centering
    \resizebox{.9\textwidth}{!}{
    \begin{tabular}{ccll}
        \toprule
        Year & Reference & Environment & Dominating Strategies\\
        \midrule
        1980 & \cite{Axelrod1980a} & Round robin tournament with 13 participants & Tit For Tat \\
        1980 & \cite{Axelrod1980b} & Round robin tournament with a probabilistic ending and 13 participants & Tit For Tat\\
        1984 & \cite{Axelrod1981}  & Evolutionary tournament with 64 participants & Tit For Tat\\
        1991 & \cite{Bendor1991}   & Round robin tournament with noise and 13 participants & Nice and Forgiving\\
        2005 & \cite{kendall2007iterated}  & Varied with 223 participants & Varying \\
        2012 & \cite{Stewart2012}  & Round robin tournament with 13 & Generous Zero Determinants \\
        2016 & \cite{Knight2016}   & Round robin tournament with 130 participants & Heuristically trained strategies\\
        2017 & \cite{Harper2017}   & Round robin tournament with 200 participants & Heuristically trained strategies\\
        \bottomrule
    \end{tabular}}
    \caption{An overview of published tournaments that introduced dominating IPD strategies in their
    respective environments. These strategies were either explicitly calculated,
    intelligently designed, or were developed through training methods.}
    \label{table:tournament_refs}
\end{table}

More details on these works will be presented in
Chapter~\ref{chapter:literature_review}, and following
Chapters~\ref{chapter:literature_review} and ~\ref{chapter:bibliometric_study}
it will become evident that the literature on the IPD is rich, and new
strategies and competitions are being published every year.
The question, however, still remains the same: what is the best way to
play the game?

\section{Research Questions \& Thesis Structure}\label{section:introduction_research_questions}

This thesis contains seven chapters, together attempting to answer three research
questions:

\begin{enumerate}
    \item What is the optimal behaviour an Iterated Prisoner's Dilemma strategy should adapt as a response to different
 environments?\label{research_question_one}
    \item What are the research topics in the field of the Prisoner's Dilemma?\label{research_question_two}
    \item Is the academic field of the Prisoner's Dilemma cooperative and is there
    influence between the authors?\label{research_question_three}
\end{enumerate}

There are two main part to the thesis. The initial part aims to answer research
questions~\ref{research_question_two} and~\ref{research_question_three}.
Chapter~\ref{chapter:literature_review} provides a condensed
literature review which summarises the already established results of the
literature. Chapter~\ref{chapter:literature_review} separates the reviewed manuscripts
under different research topics identified manually. To complement the manual
separation of articles under research topics,
Chapter~\ref{chapter:bibliometric_study} automatically partitions 2,420 IPD
articles using natural language processing. The data set of 2,420 articles' metadata has
been collected using a bespoke research software. The data set is further
analysed using network theoretic approaches to answer research question~\ref{research_question_three}.

The second part of the thesis aims to answer research
question~\ref{research_question_one}. There are four Chapters to this part which
explore optimal behaviour using original approaches. Namely,
Chapter~\ref{chapter:meta_tournaments} analyses a large data set of computer
tournaments of distinct types and evaluates \numberofstrategies strategies'
performance. Chapter~\ref{chapter:memory_one} explores best response strategies
to environments of memory-one opponents and
Chapter~\ref{chapter:best_response_sequence} explores best response strategies
in the form of sequences to a collection of opponents. Finally
Chapter~\ref{chapter:lstm}, uses the data set of best response sequences
generated in Chapter~\ref{chapter:best_response_sequence} to train an IPD
strategy using a recurrent neural network.

The seven chapters of the thesis progress in a logical manner, illustrated in
Figure~\ref{fig:structure_of_thesis}. An arrow between one chapter and another
implies that the work described in one serves as motivation for the other.

\begin{figure}[!hbtp]
    \centering
    \includestandalone[width=\textwidth]{src/chapters/01/tex/thesis_structure}
    \caption{Structure of this thesis.}\label{fig:structure_of_thesis}
\end{figure}

A summary of each chapter is outlines:

\begin{itemize}
    \item Chapter~\ref{chapter:introduction} has contextualised the three main
    research questions of this thesis. Background of game theory and the
    Prisoner's Dilemma have been given, and the structure
    of the remainder of the thesis outlined.
    \item Chapter~\ref{chapter:literature_review} provides a systematic
    literature review for the Prisoner's Dilemma and a manual classification of
    the reviewed papers under research topics. The manually identified research
    topics include evolutionary dynamics, intelligently designed strategies
    and structured strategies and training.
    \item Chapter~\ref{chapter:bibliometric_study} presents a bibliometric
    analysis of 2,420 Iterated Prisoner's Dilemma articles. It uses natural language processing to
    identify five research topics, and a graph theoretic approach to assert the
    collaborativeness of the field. The five identified topics are human subject
    research, biological studies, strategies, evolutionary dynamics on networks
    and modelling problems as a Prisoner's Dilemma.
    \item Chapter~\ref{chapter:meta_tournaments} generates and analyses a large data set of
    computer tournaments. It evaluates \numberofstrategies strategies, many of
    which are well known strategies from the literature, in 45,686 tournaments.
    It presents the top performing strategies and analyses their salient features.
    \item Chapter~\ref{chapter:memory_one} explores best responses to a
    collection of memory-one strategies as a multidimensional non linear
    optimisation problem. It presents a closed form algebraic expression for the
    utility of a memory-one strategy against a given set of opponents, a
    compact method of identifying it's best response to that given set of
    opponents whilst having a theory of mind, and it introduces a well designed
    framework that allows the comparison of an optimal memory one strategy and a
    more complex strategy which has a larger memory.
    \item Chapter~\ref{chapter:best_response_sequence} explores the problem of
    Iterated Prisoner's Dilemma best responses in the form of sequences.
    It heuristically identifies the best response sequence against 197 strategies,
    and generates a data set of 751 best response sequences of 205 turns.
    \item Chapter~\ref{chapter:lstm} uses the data set of best response strategies
    obtained from Chapter~\ref{chapter:best_response_sequence} to train a recurrent
    neural network to play the Iterated Prisoner's Dilemma.
    \item Chapter~\ref{chapter:conclusion} summarises the work of the previous
    six chapters, and indicates possible directions of future work, identifying
    further research questions that have arisen.
\end{itemize}

Chapters~\ref{chapter:meta_tournaments}-\ref{chapter:lstm} explore optimal the
behaviour in the Iterated Prisoner's Dilemma. The disparity between the
approaches is their depth, as illustrated in Figure~\ref{fig:depth_structure}.
Consider Chapter~\ref{chapter:meta_tournaments}. It explores optimal behaviour
by analysing a data set of tournaments and evaluating the performance of
pre-designed strategies. The exact opposite is done in
Chapter~\ref{chapter:best_response_sequence} whereas for a given set of two
memory-one opponents a best response strategy is calculated explicitly.
Similarly, a best responses sequence against a given opponent is calculated in
Chapter~\ref{chapter:best_response_sequence} but this is done using a heuristics
method. Finally, Chapter~\ref{chapter:lstm} using a machine learning algorithm
to generate a optimal behaviour based on recurrent networks without any manual
input.

\begin{figure}[!hbtp]
    \centering
    \includestandalone[width=.9\textwidth]{src/chapters/01/tex/depth_structure}
    \caption{The depth of exploration whilst reporting on research question 1.}\label{fig:depth_structure}
\end{figure}

\section{Software Development \& Best Practices}\label{section:introduction_software_development}

A survey conducted by the Software Sustainability Institute at 15 Russell Group
Universities showed that 92\% of the researchers questioned use software
intensively in their work, and 70\% said that ``It would not be practical to
conduct my work without software''~\cite{ssi_blog}. Similarly, the research of
this thesis heavily relies on software. As with all research there is an
obligation to ensuring the correctness and reproducibility of the results and the
software decisions throughout this thesis have been driven by these
requirements.

For the research of each chapter (excluding Chapters~\ref{chapter:literature_review}
and~\ref{chapter:conclusion}) source code and analysis code have been developed.
They have been made public via GitHub and they have one of the most flexible and
permissive licences, the MIT licence. All code of this work is written in Python, an open source
language. Essentially the code developed for the thesis is available for inspection,
testing, and modification which enables and encourages
greater understanding of the underlying methodology, increases model confidence,
and provides an extendable framework which can currently be used by others.

Two themes that arise as vital in research software development:
reproducibility and sustainability. To reassure the reproducibility and
sustainability of the software, and subsequently the research described in the
thesis, several methods of
\textit{best practice}~\cite{Aberdour2007, Benureau2018, Crick2014, Hong2015}
were considered and implemented during development. Namely:

\begin{itemize}
    \item Version control
    \item Virtual environments
    \item Automated testing
    \item Documentation
\end{itemize}

These will be discussed in the following subsections.

\subsection{Version control}

\textit{Version control}, it is a system which records all files that make up a
project (down to the line) over time, tracking their development. It also
provides the ability to recall previous versions of files. This type of system
is essential for ensuring reproducibility of scientific research~\cite{Sandve2013,
Wilson2014}.

A good version control system has the following features as stated in~\cite{Ruparelia2010}:

\begin{itemize}
    \item Backup and restore: Files can be saved as they are edited and have the facility to
    jump to a previous version.
    \item Synchronisation: Source code files can be shared and users can update their
    codebase with the latest version.
    \item Undo changes: Changes made to the code can be undone by going back
    to a version that was committed in the past.
    \item Track changes: Messages are attacked to file changes in order to track the
    how and why the code evolved over time.
    \item Track ownership: File changes are tag with the user's names who made
    the changes.
    \item Sandboxing: The ability to make temporary changes in an isolated area,
    called a sandbox, to test and try out code before it is checked in.
    \item Branching and merging: This is akin to a larger sandbox. Users can
    branch a copy of the code into a separate area and modify it in
    isolation(tracking changes separately). Later, the work can be merged back
    into the original codebase.
\end{itemize}

There are a number of popular tools for version control, these include Git~\cite{git},
Subversion~\cite{subversion}, and Mercurial~\cite{mercurial}. The version
control system chosen to carry out the software development here is Git.

There are several services that host git serves online and allow users to work
with Git publicly. These services are essential for reproducibility as they make
not only the source code for the computer programmes available online but also
the history of it's development. Such services are GitHub~\cite{github},
SourceForge~\cite{sourceforge}, Gitlab~\cite{gitlab}, and BitBucket~\cite{bitbucket}.
Github is the chosen service for the thesis which integrates well with Git.

An important feature of GitHub is that it fosters collaboration between users.
It is a social service which allows user to comment and raise issues on each
other's repositories. Moreover, it encourages collaboration and code contributions
by other users through pull request features. An example of pull request on
a GitHub repository whereas the history of the development is viewable in the
form of a commit is given by Figure~\ref{fig:pull_request_github}.
In Figure~\ref{fig:pull_request_github} it is also illustrated how issues can be
addressed and closed from a specific pull request.

\begin{figure}[!hbtp]
    \centering
    \includegraphics[width=.8\textwidth]{src/chapters/01/img/GitHub_discussion}
    \caption{An example of a pull request on GitHub.}\label{fig:pull_request_github}
\end{figure}

The code for Chapters'~\ref{chapter:bibliometric_study}-\ref{chapter:lstm} is
hosted on individual GitHub repositories. Details of the repositories are
given by Table~\ref{table:source_code_data_citations}. The source code for each
repository has been packaged and is also archived on Zenodo~\cite{zenodo}.

\subsection{Virtual environments}

There are GitHub repositories corresponding to each of the chapters of the thesis.
The source code associated with each repository makes use of several Python
libraries. Though several of these projects use the same libraries the versions
of these libraries can differ.

The tool used here to keep dependencies required by the different projects
separate is creating isolated Python \textit{virtual environments} for them. The
virtual environments used hare are Anaconda environments which integrate easily
with the programming language Python. Anaconda is a free and open-source
distribution of the Python and R programming languages for scientific computing,
that aims to simplify package management and deployment. Package versions are
managed by the package management system conda.

Conda allows users to create, export, list, remove, and update environments that
have different versions of Python and/or packages installed in them. Switching
or moving between environments is called activating the environment. An environment
can be shared and kept under version control as a file. An example of
such as file is given by Figure~\ref{fig:environment_file}.

\begin{figure}
\begin{shell}
name: opt-mo
channels:
  - defaults
dependencies:
  - python=3.6.7
  - numpy=1.15.4
  - pandas=0.23.4
  - pip:
    - attrs==19.1.0
    - axelrod==4.4.0
    - black==18.9b0
    - sympy==1.2.0
    - scikit-optimize==0.5.2
    - jupyter==1.0.0
    - jupyter-console==5.2.0
    - ipython==6.4.0
    - pytest==4.0.1
    - pytest-cov==2.7.1
    - sqlalchemy==1.2.17
    - fsspec==0.3.3
\end{shell}
\caption{An example of an environment file. The name of the specific environment is called
\mintinline{shell}{opt-mo} and it corresponds to the environment associated with
Chapter~\ref{chapter:memory_one}.}\label{fig:environment_file}
\end{figure}

Each repository of the thesis include an environment file detailing the dependencies
of the source code and their versions.

\subsection{Automated testing}

Testing code is of considerable importance in order to ensure the robustness,
correctness and sustainability of the computer code. The standard method of testing code is through
\textit{automated testing} using test suits that run parts of the code and
assert whether they are behaving as expected.

Two types of tests are described in~\cite{Percival2014}, \textit{functional
tests} that assert the code's functionality, and \textit{unit tests} that help
ensure the code is clean and free of bugs.

Functional tests aim to test how the whole application functions from the
perspective of the outside user. They feed in basic input and test whether the
end product/final behaviour is as expected. Unit tests assert that small chunks
of code behave as expected, and the test the application from the point of the
programmer. Unit tests are isolated from the rest of the code and are modular.
There are two types of unit tests: \textit{pure} and \textit{integrated} tests.

Pure unit tests are written to test only one function or method. Thus, if a pure
unit test was to fail then it should be due to problems with the specific part
of the code it is testing only, and not any other bit of code. Pure unit tests
are fast and readable, however they do not test how well functions and methods
integrate with one another. In order to test these, unit tests must be written
that rely on other parts of the code that aren't explicitly being tested. This
type of test is called integrated tests.

The automated tests, which include functional and unit tests, of the
repositories associated with the thesis have been implemented using the Python
library \mintinline{python}{pytest}. The \mintinline{shell}{pytest} framework
makes it easy to write automated tests in a few lines. A plugin to
\mintinline{shell}{pytest} is that of \mintinline{shell}{pytest-cov}. The plugin
allows for coverage checks, to test how much of the code is covered by the testing
suite.

To regularly test the code that is being push to the GitHub repositories a
continuous integration (CI) system is used. CIs run the tests suite of a
repository every time a push happens. The benefits of using a CI are identifying
bugs quickly, reducing problems when merging in contributions from
collaborators, and adding transparency to the development process. There are two
CIs that have been used in the repositories listed in
Table~\ref{table:source_code_data_citations}. These are Travis~\cite{travis} and
GitHub Actions~\cite{github_actions}. Both integrate with GitHub.

\subsection{Documentation}

Software \textit{documentation} is written text or illustration that accompanies
computer software or is embedded in the source code. The documentation either
explains how the software operates or how to use it.

Each repository associated with the thesis includes a detailed README file.
These contain installation instructions for corresponding packaged source code
and demonstrate how to run the associated test suite. The source code for each
repository has been written in a modular way and meaningful names have been
given to all variables, functions, methods and classes. Each function, method
and class includes a \text{docstring}. A docstring is a series of sentences used
to document a specific segment of code.

The repositories also include a series of Jupyter Notebooks~\cite{jupyter} that
are used to carry out the analysis of each Chapter, and serve as demonstration
of the source code's usage.

\subsection{Thesis's Code}

As previously stated the source codes for
Chapters~\ref{chapter:bibliometric_study}-\ref{chapter:lstm} have been written
following best practices, have been packaged, are available on GitHub and has
been archived on Zenodo. These practices have been followed to reassure the
the correctness, reproducibility and sustainability of the research
the research described throughout the thesis. For reassuring the reproducibility
of the work the data sets used in several of following Chapters have also
been archived and are available online. The details for the source code and
data sets for each Chapter are summarised in Table~\ref{table:source_code_data_citations}.

\begin{table}[htbp]
    \centering
    \resizebox{\textwidth}{!}{
    \begin{tabular}{llcc}
        \toprule
        {} & GitHub url & Source code citation & Data citation \\
        \midrule
        Chapter~\ref{chapter:bibliometric_study}    & \url{https://github.com/Nikoleta-v3/bibliometric-study-of-the-prisoners-dilemma} &
        \cite{nikoleta_2017} &  \cite{auction_data_2018, anarchy_data_2018, pd_data_2018}\\
        Chapter~\ref{chapter:meta_tournaments}      & \url{https://github.com/Nikoleta-v3/meta-analysis-of-prisoners-dilemma-tournaments} &
        & \cite{data} \\
        Chapter~\ref{chapter:memory_one}            & \url{https://github.com/Nikoleta-v3/Memory-size-in-the-prisoners-dilemma}  &  & \cite{Glynatsi2019} \\
        Chapter~\ref{chapter:best_response_sequence}
        \&~\ref{chapter:lstm}                       & \url{https://github.com/Nikoleta-v3/Training-IPD-strategies-with-RNN}   &  & \cite{Glynatsi2020} \\
        \bottomrule
    \end{tabular}}
    \caption{Citations and GitHub url for source code and data used in the thesis.}
    \label{table:source_code_data_citations}
\end{table}

Through the thesis parts of the source code and demonstration of it's usage are
going to be mentioned at the corresponding chapters.

Several chapter of the thesis make use of the open source package
Axelrod-Python. Axelrod-Python is an open source project for simulating rounds
of the IPD. The specific version of Axelrod-Python used at each chapter will be
mentioned at the start of each chapter.

This thesis itself is hosted on a GitHub repository:
\url{https://github.com/Nikoleta-v3/Thesis}. The thesis is written in \LaTeX and
there automated tests have been setup to tests that the document compiles each
time there is a push, and that not words in the document have been misspelled.