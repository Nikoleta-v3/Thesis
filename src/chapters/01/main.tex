\chapter{Introduction}

This thesis aim to reinforce the understanding of best responses to environments
for the Prisoner's Dilemma, one the the most well known examples of a strategic
game. In the 1984's the book ``The Evolution of Cooperation'' was an initial
step to computer science being applied to  a field of mathematics used to
understand the emergence of cooperative behaviour.

This thesis summarises, evaluates and builds upon previous literature to
push the bounds of our understanding of how mathematical modelling and
computer science techniques can used in the search of dominant
behaviour to environments. To summarise, this thesis contributes to the discussion
of dominant behaviour and provides new mathematical frameworks for the continued
understanding of the questions raised throughout.

This introductory chapter is set as follows:

\begin{itemize}
    \item Section~\ref{section:introduction_prisoners_dilemma} introduces
    the Prisoner's Dilemma.
    \item Section~\ref{section:introduction_brief_literature} covers a brief
    literature review.
    \item Section~\ref{section:introduction_research_questions} formalises the research
    questions and sets out the structure of the thesis.
    \item Section~\ref{section:introduction_software_development} cover the
    software development techniques used throughout of the thesis.
\end{itemize}

\section{Prisoner's Dilemma}\label{section:introduction_prisoners_dilemma}

Game theory was formalised in 1944~\cite{VonNeumann1944} and is the study of
interactions as \textit{strategic games}. A game is a model of interacting
decision makers refereed to as \textit{players}. Each player has a set of
possible \textit{actions}. The model captures the interactions of the player's
actions by allowing each player's \textit{payoffs} to be affected by the actions
of all players. More precisely, as given in~\cite{Osborne2004}, the formal
definition of a strategic game is as follows,

\begin{definition}
A \textbf{strategic game} consists of
\begin{itemize}
    \item a set of player
    \item for each player, a set of actions
    \item for each player, payoff functions over the set of all actions.
\end{itemize}
\end{definition}

On the most well known strategic games is the Prisoner's Dilemma (PD) originally
described in~\cite{Flood1958}. A two player non-cooperative game which
illustrates aspects of political philosophy and morality; how selfishness will
lead to an `inefficiency' of the outcome, when selflessness can be
evolutionarily advantageous.

More specifically, in the PD each player has two actions, to either be selfless
and cooperate, denoted as (C), or to be selfish and defect, denoted as (D). Each
decision is made simultaneously and independently. Players' payoffs are
generally represented by (\ref{eq:pd_definition}). Both players receive a reward
for mutual cooperation, \(R\), and a payoff \(P\) for mutual defection. A player
that defects while the other cooperates receives a payoff of \(T\), whereas the
cooperator receives \(S\).

\begin{equation}\label{eq:pd_definition}
    S_p =
    \begin{pmatrix}
        R & S  \\
        T & P
    \end{pmatrix}
    \quad
    S_q =
    \begin{pmatrix}
        R & T  \\
        S & P
    \end{pmatrix}
\end{equation}

It is assumed that two cooperating players do better than two defecting ones,
and thus, the payoff of two cooperating players \(R\) is
larger than the payoff of two defecting players \(P\). A player, however, has the
temptation to deviate, as that player will receive a higher payoff \(T\) than
that of mutual cooperation \(R\) whilst the cooperator's payoff \(S\) is smaller than
\(P\). In consequence, the payoffs are constrained by
Equation~(\ref{eq:constrain_one}).

\begin{equation}\label{eq:constrain_one}
    T > R > P > S
\end{equation}

A second constraint is that which ensures that a social dilemma arises: the sum
of the utilities to both players is best when they both cooperate,
Equation~(\ref{eq:constrain_two}).

\begin{equation}\label{eq:constrain_two}
    2R > T + S
\end{equation}

A special case of the Prisoner's Dilemma is that of the \textit{donation game}.
In the donation game each player can cooperate by providing a benefit \(b\) to
the other player at their cost \(c\) with \(0 < c < d\). Then \(T=b, R=b-c,S=c,
\text{and } P=0\). The representation of the donation game payoff matrix is
given by~(\ref{eq:the_pd_payoffs_with_cost}).

\begin{equation}\label{eq:the_pd_payoffs_with_cost}
    S_p =
    \begin{pmatrix}
        b - c & c\\
        b & 0
    \end{pmatrix}
    \quad
    S_q =
    \begin{pmatrix}
        b - c & b  \\
        c & 0
    \end{pmatrix}
\end{equation}

This thesis uses the basic PD as given by Equation~\ref{eq:pd_definition}
throughout, and the values of the payoff used are \(R = 3,P = 1,T = 5,S = 0\).
These values are the most common in the literature.

In a non cooperative game the players interact in order to achieved a most
preferred outcome. The \textit{best response strategy} is the strategy
that maximises the utility of a player given a known strategy of the other
player. A solution concept commonly used in game theory is the Nash equilibrium
which is a pair of best response strategies at which neither of the players has
a reason to deviate. Due to constrains (\ref{eq:constrain_one} and
\ref{eq:constrain_two}) it never benefits a player to cooperate, and once both
players defect neither have a reason to change their decision. Thus, in the
Prisoner Dilemma mutual defection is a Nash equilibrium and defection is the
best response strategy.

The game can be studied in a manner where prior outcome matters whereas
defecting is no longer necessarily. The repeated form of the game is called the
Iterated Prisoner's Dilemma (IPD). The IPD differs from the original concept of
a PD because participants can learn about the behavioral tendencies of their
counter party. The dominant choice and the best response strategies are no
longer trivial. Theoretical work has shown that once players interact repeatedly
complex behaviour can emerge. Arguably, the most important of these works is
Robert Axelrod's ``The Evolution of Cooperation''~\cite{Axelrod1984} where he
reports on a series of computer tournaments. These demonstrated that in repeated
interactions reciprocity allows cooperative behaviour to emergence.

\section{Brief Literature Review}\label{section:introduction_brief_literature}

In the 1980s, Axelrod published on an evolutionary tournament~\cite{Axelrod1981}
and two round robin tournaments~\cite{Axelrod1980a, Axelrod1980b}. The results
highlighted the robustness the strategy Tit For Tat which was the winner
of both round robin tournaments and took over a population of 63 strategies.
The strategy has kept it's reputation until today and it can be commonly be
consider as the best basic strategy in the IPD.

Nevertheless, following Axelrod's tournaments new competitions and mathematical
models introduced new strategies, each with new traits of dominance.
A brief summary of selected competitions and the dominating strategies that risen
from these are given by Table~\ref{table:tournament_refs}.

\begin{table}[htbp]
    \centering
    \resizebox{.9\textwidth}{!}{
    \begin{tabular}{ccll}
        \toprule
        Year & Reference & Environment & Dominating Strategies\\
        \midrule
        1980 & \cite{Axelrod1980a} & Round robin tournament with 13 participants & Tit For Tat \\
        1980 & \cite{Axelrod1980b} & Round robin tournament with a probabilistic ending and 13 participants & Tit For Tat\\
        1984 & \cite{Axelrod1981}  & Evolutionary tournament with 64 participants & Tit For Tat\\
        1991 & \cite{Bendor1991}   & Round robin tournament with noise and 13 participants & Nice and Forgiving\\
        2005 & \cite{Kendall2007iterated}  & Varied with 223 participants & Varying \\
        2012 & \cite{Stewart2012}  & Round robin tournament with 13 & Generous Zero Determinants \\
        2012 & \cite{Harper2017}   & Round robin tournament with 200 participants & Heuristically trained strategies\\
        \bottomrule
    \end{tabular}}
    \caption{An overview of published tournaments}
    \label{table:tournament_refs}
\end{table}

\section{Research Questions \& Thesis Structure}\label{section:introduction_research_questions}

This thesis contains seven chapters, together attempting to answer two research
questions:

\begin{enumerate}
    \item What is the optimal behaviour an Iterated Prisoner's Dilemma strategy should adapt as a response to different
 environments?
    \item What is are the research topics in the field of the Prisoner's Dilemma?
    \item Is the academic field of the Prisoner's Dilemma cooperative and is there
    influence between the authors?
\end{enumerate}

There are two main part to the thesis. The initial parts aims to answer research
questions 2 and 3. Chapter~\ref{chapter:literature_review} provides a condensed
literature review which summarises the already established results of the
literature. Chapter~\ref{chapter:literature_review} separates the manuscripts
under different research topics identified by myself. To complement the manual
separation of articles under research topics,
Chapter~\ref{chapter:bibliometric_study} automatically partitions 2500 IPD
articles using natural language processing. The data set with 25000 articles have
been collected using a bespoke research software. The data set is further
analysed using network theoretic approaches to answer research question 3.

The second part includes four chapter and it aims to answer research question 1.
Each chapter uses a different approach, namely:
Chapter~\ref{chapter:meta_tournaments} evaluates the performance of 195, many
from the literature, in more than 40000 of different types.
Chapter~\ref{chapter:memory_one}, focuses on a specific set of IPD strategies
called memory-one. The thesis uses mathematical modelling to explore the best
responses to a collection of memory-one strategies as a multidimensional non
linear optimisation problem, and the benefits of extortionate/manipulative
behaviour. Chapter~\ref{chapter:best_response_sequence} explores best responses
in the form of sequences and identifies a series of best response sequences to
strategies from the literature heuristically. Finally,
Chapter~\ref{chapter:lstm} trains a sophisticated strategy using a recurrent neural
network.

The difference between these chapters is their depth in which the explore best
responses. More specifically, \dots as shown in Figure~\ref{fig:depth_structure}.

\begin{figure}[!hbtp]
    \centering
    \includestandalone[width=\textwidth]{src/chapters/01/depth_structure}
    \caption{The depth of exploration whilst reporting on research question 1.}\label{fig:depth_structure}
\end{figure}

The seven chapters of the thesis progress in a logical manner, illustrated in
Figure~\ref{fig:structure_of_thesis}. An arrow between one chapter and another
implies that work from one is used as motivation of the other.

\begin{figure}[!hbtp]
    \centering
    \includestandalone[width=.9\textwidth]{src/chapters/01/thesis_structure}
    \caption{Structure of this thesis.}\label{fig:structure_of_thesis}
\end{figure}

A summary of each chapter is outlines:

\begin{itemize}
    \item Chapter~\ref{chapter:literature_review} provides a systematic
    literature review for the Prisoner's Dilemma. It has manually classified
    paper under research topics and.
    \item Chapter~\ref{chapter:bibliometric_study} provides a systematic
    literature review for the Prisoner's Dilemma. It has manually classified
    paper under research topics and 
    \item Chapter~\ref{chapter:meta_tournaments} analyses a large data set of
    computer tournaments. It evaluates \numberofstrategies strategies in more
    than 40000 tournaments, presents the top performing strategies, and analyzes
    their salient features.
    \item Chapter~\ref{chapter:memory_one} explores the full space of memory-one
    strategies and identifies the best response memory-one strategy against a
    given set of opponents.
    \item Chapter~\ref{chapter:best_response_sequence}
    \item Chapter~\ref{chapter:lstm}
    \item Chapter 8 summarises the work and indicates possible future directions
    for the work.
\end{itemize}

\section{Software Development \& Best Practices}\label{section:introduction_software_development}

The research of this thesis, as many others, heavily rely on software.
A survey conducted by the at 15 Russell Group Universities.
92\% of the researchers questioned use software intensively in 
their work, and 70\% said that ``It would not be practical to 
conduct my work without software''.

As with all research there is an obligation to ensuring the correctness and
reproducibility of the results. The software decisions throughout this thesis
have been driven by these requirements.

Each chapter of this thesis, excluding Chapters~\ref{chapter:literature_review}
and~\ref{chapter:conclusion} have source code and analysis code associated with
them. The code of this thesis is written in the programming language Python. The
source code, the analysis notebooks and the data for each chapter have been
archived and are available online. The source code and the analysis notebooks
are also hosted on GitHub. More specifically,

\begin{table}[htbp]
    \centering
    \resizebox{\textwidth}{!}{
    \begin{tabular}{llcc}
        \toprule
        {} & GitHub url & Source code citation & Data citation \\
        \midrule
        Chapter~\ref{chapter:bibliometric_study}    & \url{https://github.com/Nikoleta-v3/bibliometric-study-of-the-prisoners-dilemma} &
        \cite{Nikoleta_2017} &  \cite{Pd_data_2018, Anarchy_data_2018, Auction_data_2018}\\
        Chapter~\ref{chapter:meta_tournaments}      & \url{https://github.com/Nikoleta-v3/meta-analysis-of-prisoners-dilemma-tournaments} &
        & \cite{Data} \\
        Chapter~\ref{chapter:memory_one}            & \url{https://github.com/Nikoleta-v3/Memory-size-in-the-prisoners-dilemma}  &  & \cite{Glynatsi2019} \\
        Chapter~\ref{chapter:best_response_sequence}
        \&~\ref{chapter:lstm}                       & \url{https://github.com/Nikoleta-v3/Training-IPD-strategies-with-RNN}   &  & \\
        \bottomrule
    \end{tabular}}
    \caption{Citations and GitHub url for source code and data used in the thesis.}
    \label{table:source_code_data_citations}
\end{table}

These are also mentioned at the corresponding chapters alongside examples of
the code's usage.

Several chapter of the thesis make use of the open source package Axelrod-Python.
Axelrod-Python is an open source project for simulating rounds of the IPD.
The specific version of Axelrod-Python used at each chapter will be mentioned
at the start of each chapter.

This thesis itself is hosted on GitHub and uses automatic testing \dots.