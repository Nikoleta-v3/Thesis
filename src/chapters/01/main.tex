\chapter{Introduction}

This thesis describes ...

This introductory chapter is set as follows:

\begin{itemize}
    \item Section~\ref{section:introduction_prisoners_dilemma} introduces
    the Prisoner's Dilemma.
    \item Section~\ref{section:introduction_brief_literature} covers a brief
    literature review.
    \item Section~\ref{section:research_questions} formalises the research
    questions and sets out the structure of the thesis.
\end{itemize}

\section{Prisoner's Dilemma}\label{section:introduction_prisoners_dilemma}

Based on the Darwinian principle of survival of the fittest cooperative behaviour
should not be favoured, however, cooperation is plentiful in nature.
A paradigm of understanding the emergence of these behaviours is
a particular two player non-cooperative game called the Prisoner's Dilemma (PD),
originally described in~\cite{Flood1958}.

In the PD each player has two choices, to either be selfless and cooperate or to
be selfish and defect. Each decision is made simultaneously and
independently. The utility of each player is influenced by its own behaviour,
and the behaviour of the opponent. Both players do better if they choose to
cooperate than if both choose to defect. However, a player has the temptation to
deviate as that player will receive a higher payoff than that of mutual
cooperation.
Players' payoffs are generally represented by (\ref{eq:the_pd_payoffs}). Both
players receive a reward for mutual cooperation, \(R\), and a payoff \(P\) for
mutual defection. A player that defects while the other cooperates receives a payoff of
\(T\), whereas the cooperator receives \(S\). The dilemma exists due
to constraints (\ref{eq:constrain_one}) and (\ref{eq:constrain_two}).

\begin{equation}\label{eq:the_pd_payoffs}
    \begin{pmatrix}
    R & S \\ T & P
    \end{pmatrix}
\end{equation}

\begin{equation}\label{eq:constrain_one}
    T > R > P > S
\end{equation}

\begin{equation}\label{eq:constrain_two}
    2R > T + S
\end{equation}

Another common representation of the payoff matrix is given by~(\ref{eq:the_pd_payoffs_with_cost}),
where \(b\) is the benefit of the altruistic behaviour and \(c\) it's its cost
(constraints (\ref{eq:constrain_one}) and (\ref{eq:constrain_two}) still hold).

\begin{equation}\label{eq:the_pd_payoffs_with_cost}
    \begin{pmatrix}
        b - c & c \\ b & 0
    \end{pmatrix}
\end{equation}

Constraints (\ref{eq:constrain_one}-\ref{eq:constrain_two})
guarantee that it never benefits a player to cooperate, indeed mutual
defection is a Nash equilibrium. However, when the game is studied in a manner
where prior outcome matters, defecting is no longer necessarily the dominant
choice.

The repeated form of the game is called the Iterated Prisoner's Dilemma (IPD)
and theoretical works have shown that cooperation can emerge once players
interact repeatedly. Arguably, the most important of these works is Robert
Axelrod's ``The Evolution of Cooperation''~\cite{Axelrod1984}. In his book
Axelrod reports on a series of computer tournaments he organised. In these
tournaments academics from several fields were invited to design computer
strategies to compete. Axelrod's work showed that greedy
strategies did very poorly in the long run whereas altruistic strategies did
better. ``The Evolution of Cooperation'' is considered a milestone in the field
but it is not the only one. On the contrary, the PD has attracted attention ever
since the game's origins.


\section{Brief Literature Review}\label{section:introduction_brief_literature}

\section{Research Questions \& Thesis Structure}\label{section:research_questions}

This thesis contains ... chapters, together attempting to answer three research
questions.

\section{Software Development \& Best Practices}
