\chapter{Abstract}

This thesis investigates the optimal behaviour that Iterated Prisoner's Dilemma
strategies should adopt as a response to different environments. The Iterated
Prisoner's Dilemma (IPD) is a particular topic of game theory that has attracted
academic attention due to its applications in the understanding of
the balance between cooperation and competition in social and biological
settings.

This thesis uses a variety of mathematical and computational fields such as
linear algebra, research software engineering, data mining, network
theory, natural language processing, data analysis, mathematical
optimisation, resultant theory, markov modelling, agent based simulation,
heuristics and machine learning.

The literature around the IPD has been exploring the performance of strategies
in the game for years. The results of this thesis contribute to the discussion
of successful performances using various novel approaches.

Initially, this thesis evaluates the performance of \numberofstrategies
strategies in \numberofalltournaments computer tournaments. A large portion of
the \numberofstrategies strategies are drawn from the known and named strategies in the IPD
literature, including many previous tournament winners. The \numberofalltournaments
computer tournaments include tournament variations such as tournaments with
noise, probabilistic match length, and both noise and probabilistic match
length. This diversity of strategies and tournament types has resulted in the largest and
most diverse collection of computer tournaments in the field. The impact of
features on the performance of the \numberofstrategies strategies is evaluated
using modern machine learning and statistical techniques. The results reinforce
the idea that there are properties associated with success, these are: be nice,
be provocable and generous, be a little envious, be clever, and adapt to the
environment.

Secondly, this thesis explores optimal behaviours focused on a specific set of
IPD strategies called memory-one, and specifically a subset of them that are
considered extortionate. These strategies have gained much attention in the
research field and have been acclaimed for their performance against single
opponents. This thesis uses  mathematical modelling to explore the best
responses to a collection of memory-one strategies as a multidimensional non-linear
optimisation problem, and the benefits of extortionate/manipulative
behaviour. The results contribute to the discussion that behaving in an
extortionate way is not the optimal play in the IPD, and provide evidence that
memory-one strategies suffer from their limited memory in multi agent
interactions and can be out performed by longer memory strategies.

Following this, the thesis investigates best response strategies in the form of
static sequences of moves. It introduces an evolutionary algorithm which can
successfully identify best response sequences, and uses a list of
\numberofstrategiesbestsequences opponents to generate a large data set of best
response sequences. This data set is then used to train a type of recurrent
neural network called the long short-term memory network, which have not gained
much attention in the literature. A number of long short-term memory networks
are trained to predict the actions of the best response sequences. The trained
networks are used to introduce a total of \lstmstrategies new IPD strategies which were shown
to successfully win standard tournaments.

From this research the following conclusions are made: there is not a single
best strategy in the IPD for varying environments, however, there are properties
associated with the strategies' success distinct to different environments.
These properties reinforce and contradict well established results. They include
being nice, opening with cooperation, being a little envious, being complex,
adapting to the environment and using longer memory when possible.