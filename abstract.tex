\chapter{Abstract}

This thesis investigates the optimal behaviour that Iterated Prisoner's Dilemma
strategies' should adapt as a response to different environments. The Iterated
Prisoner's Dilemma is a particular topic of game theory that has attracted the
academic attention. This is due to its applications in the understanding of the
balance between cooperation and competition in social and biological settings.

This thesis refers to a variety of mathematical and computational fields such as
mathematical formulation, research software engineering, data mining, network
theory, natural language processing, data analysis, big data, mathematical
optimisation, resultant theory, markov modelling, agent based simulation,
heuristics, machine learning and deep learning. %TODO does this sentence adds value

In the Iterated Prisoner's Dilemma the literature has been exploring the
performance of strategies in the game for years. The results of this thesis
contribute to the discussion of successful performances using various novel
approaches.

Initially, this thesis evaluates the performance of strategies in a meta
tournament analysis. The results of the literature, have been relying on a small
number of somewhat arbitrarily selected strategies in a small number of
tournaments, whereas this thesis evaluates \numberofstrategies strategies'
effectiveness in \numberofalltournaments. A large portion of these strategies
are drawn from the known and named strategies in IPD literature, including many
previous tournament winners. The analysis also considers tournament variations
including standard tournaments, tournaments with noise, probabilistic match
length, and both noise and probabilistic match length. This diversity of
strategies and tournament types yields the largest and most diverse collection
of computer tournaments in the literature. The thesis also evaluates the impact
of features on the performance of the  \numberofstrategies strategies using
modern machine learning and statistical techniques. The results reinforce the
idea that there are properties associated with success, these are: be nice, be
provocable and generous, be a little envious, be clever, and adapt to the
environment.

Secondly, this thesis explores optimal behaviours focused on a specific set of
Iterated Prisoner's Dilemma strategies called memory-one, and specifically an
extortionate set of them. These strategies have gained much attention in the
research field and have been acclaimed for their performance against single
opponents. This thesis uses  mathematical modelling to explore the best
responses to a collection of memory-one strategies as a multidimensional non
linear optimisation problem, and the benefits of extortionate/manipulative
behaviour. The results contribute to the discussion that behaving in an
extortionate way is not the optimal play in the IPD, and provided evidence that
memory-one strategies suffer from their limited memory in multi agent
interactions and can be out performed by longer memory strategies.

Furthermore, this thesis explores best response strategies in the form of static
sequences of moves. It introduces an evolutionary algorithm which can
successfully identify best response sequences, and uses a list of
\numberofstrategiesbestsequences opponents to generate a large data set of best
response sequences. This data set is then used to train a type of recurrent
neural network called the . The trained networks are used to introduce a total
of 24 strategies which were shown to be able to win standard tournaments.

From the research conducted this doctoral research the following conclusions are
made. Though there is not a single best strategy in the Iterated Prisoner's
Dilemma for varying environments there are properties associated with the
strategies' successful distinct to different environments. These properties
reinforce and contradict well established results. The properties include
being nice, ...., having larger memory. 

Finally, an attempt was made to train a complex strategies based on a set of
recurrent neural networks. The research of the new strategy returned that for
specific topologies an accomplished strategy exists. These specific type of
strategies demonstrate that opening with cooperation is beneficial, are
never the first one to defect and.
